\documentclass{revtex4-2}
\usepackage{amsmath}
\begin{document}
\title{URG analysis of the extended Kondo model with d-wave interaction}
\author{Abhirup Mukherjee, Siddhartha Lal}
\maketitle
\section{Hamiltonian}
We consider an impurity spin \(\vec S_d\) interacting with a two-dimensional tight-binding conduction bath through the d-wave channel :
\begin{equation}\begin{aligned}
	J \vec{S}_d\cdot\vec{S_f},\quad \vec S_f = \frac{1}{2}\sum_{\sigma,\sigma^\prime}\vec \tau_{\sigma,\sigma^\prime}f^\dagger_{\sigma}f_{\sigma^\prime}.
\end{aligned}\end{equation}
where \(\vec \tau\) is the vector of sigma matrices. The spin \(\vec S_f\) is constructed in terms of the d-wave electron \(f_{\sigma} = \frac{1}{2}\left(c^\dagger_{L,\sigma} + c^\dagger_{R,\sigma} - c^\dagger_{U\sigma} - c^\dagger_{D\sigma}\right) \), where \(L,R,U\) and \(D\) indicate electrons at the positions \((x,y)=\left(-a,0\right), \left( a,0 \right) , \left( 0,a \right) \) and \(\left( 0,-a \right) \) respectively, \(a\) being the lattice spacing of the conduction bath lattice. We also consider local bath correlations in the \(d-\)wave channel:
\begin{equation}\begin{aligned}
	-\frac{W}{2}\left(f^\dagger_{\uparrow}f_{\uparrow} - f^\dagger_{\downarrow}f_{\downarrow}\right)^2.
\end{aligned}\end{equation}
In order to facilitate a momentum-space decoupling RG scheme, we Fourier transform the \(d-\)wave operator to momentum space:
\begin{equation}\begin{aligned}
	f_{\sigma} = \frac{1}{2}\left(c^\dagger_{L,\sigma} + c^\dagger_{R,\sigma} - c^\dagger_{U\sigma} - c^\dagger_{D\sigma}\right) = \frac{1}{2}\sum_{\vec k}c^\dagger_{\vec k,\sigma}\left[e^{-ik^x a} + e^{ik^x a} - e^{-ik^y a} - e^{ik^y a}\right] =\sum_{\vec k}\left[\cos\left( ak^x \right) - \cos\left( ak^y \right) \right] c^\dagger_{\vec k,\sigma}~.
\end{aligned}\end{equation}
The Kondo interaction term takes the following form in momentum space:
\begin{equation}\begin{aligned}
	J \vec{S}_d\cdot\vec{S_f} = \frac{1}{2}J \sum_{\vec k_1, \vec k_2, \sigma,\sigma^\prime}\vec{S}_d\cdot\vec{\tau}_{\sigma,\sigma^\prime}c^\dagger_{\vec k_1\sigma}c_{\vec k_2,\sigma^\prime}\prod_{i=1,2}\left[\cos\left( ak_i^x \right) - \cos\left( ak_i^y \right) \right]~.
\end{aligned}\end{equation}
The local correlation term can be similarly written in momentum space:
\begin{equation}\begin{aligned}
	-\frac{W}{2}\left(f^\dagger_{\uparrow}f_{\uparrow} - f^\dagger_{\downarrow}f_{\downarrow}\right)^2 = -\frac{W}{2}\sum_{\vec k_1,\vec k_2, \vec k_3, \vec k_4,\sigma}\left(c^\dagger_{\vec k_1,\sigma}c_{\vec k_2,\sigma}c^\dagger_{\vec k_3,\sigma}c_{\vec k_4,\sigma} - c^\dagger_{\vec k_1\sigma}c_{\vec k_2,\sigma}c^\dagger_{\vec k_3\bar\sigma}c_{\vec k_4,\bar\sigma}\right)\prod_{i=1,2,3,4}\left[\cos\left( ak_i^x \right) - \cos\left( ak_i^y \right) \right]~.
\end{aligned}\end{equation}
Combining all the terms, the Hamiltonian can be formally written as
\begin{equation}\begin{aligned}
	H = \sum_{\vec k,\sigma}\varepsilon_{\vec k}c^\dagger_{\vec k,\sigma}c_{\vec k,\sigma} + \frac{1}{2}\sum_{\vec k_1, \vec k_2, \sigma,\sigma^\prime}J_{\vec k_1, \vec k_2}\vec{S}_d\cdot\vec{\tau}_{\sigma,\sigma^\prime}c^\dagger_{\vec k_1\sigma}c_{\vec k_2,\sigma^\prime} - \frac{1}{2}\sum_{\vec k_1,\vec k_2, \vec k_3, \vec k_4,\sigma,\sigma^\prime}W_{\vec k_1,\vec k_2,\vec k_3,\vec k_4} \sigma\sigma^\prime c^\dagger_{\vec k_1,\sigma}c_{\vec k_2,\sigma}c^\dagger_{\vec k_3,\sigma^\prime}c_{\vec k_4,\sigma^\prime}~.
\end{aligned}\end{equation}
with 
\begin{equation}\begin{aligned}
	\varepsilon_{\vec k} &= -2t\left[\cos\left(ak^x\right) + \cos\left(ak^y\right)\right], \\
	J_{\vec k_1, \vec k_2} &= J\prod_{i=1,2}\left[\cos\left( ak_i^x \right) - \cos\left( ak_i^y \right) \right], \\
	W_{\vec k_1,\vec k_2,\vec k_3,\vec k_4}  &= W\prod_{i=1,2,3,4}\left[\cos\left( ak_i^x \right) - \cos\left( ak_i^y \right) \right].
\end{aligned}\end{equation}
The coupling \(W_{\vec k_1,\vec k_2,\vec k_3,\vec k_4}\) has certain symmetries. It is independent of the sequence of momentum indices. It remains invariant if any number of the momenta undergo rotation by an integer multiple of \(\pi/2\). It also remains invariant if an even number of momenta are reflected about the nodal point in the corresponding quadrant.

\section{RG scheme}
At any given step \(j\) of the RG procedure, we decouple the states \(\left\{ \vec q \right\} \) on the isoenergetic surface of energy \(\varepsilon_j\). The diagonal Hamiltonian \(H_D\) for this step consists of all terms that do not change the occupancy of the states \(\left\{\vec q\right\}\):
\begin{equation}\begin{aligned}
	H_D^{(j)} = \varepsilon_j\sum_{q,\sigma}\tau_{q,\sigma} + \sum_{\vec q}J_{\vec q, \vec q}S_d^z\left(\hat n_{\vec q, \uparrow} - \hat n_{\vec q, \downarrow}\right) - \frac{1}{2}\sum_{\vec q_1}W_{\vec q_1, \vec q_1, \vec q_1, \vec q_1}\left(\hat n_{\vec q_1, \uparrow} - \hat n_{\vec q_1, \downarrow}\right)^2~.
\end{aligned}\end{equation}
where \(\tau = \hat n - 1/2\). The three terms, respectively, are the kinetic energy of the momentum states on the isoenergetic shell that we are decoupling, the Ising interaction energy between the impurity spin and the spins formed by these momentum states and, finally, the local correlation energy associated with these states arising from the \(W\) term.

The off-diagonal part of the Hamiltonian on the other hand leads to scattering in the states \(\left\{ \vec q \right\} \):
\begin{equation}\begin{aligned}
	H_X^{(j)} =& \underbrace{\sum_{\vec k, \vec q, \sigma,\sigma^\prime}J_{\vec k, \vec q} \vec{S}_d\cdot\vec{\tau}_{\sigma,\sigma^\prime}\left[c^\dagger_{\vec q\sigma}c_{\vec k,\sigma} + \text{h.c.}\right]}_{T_1^\dagger + T_1} \\
		   &\underbrace{- \frac{1}{2}\sum_{\vec q_1,\vec k_2, \vec k_3, \vec k_4,\sigma,\sigma^\prime}W_{\vec q_1,\vec k_2,\vec k_3,\vec k_4} \sigma\sigma^\prime \left[c^\dagger_{\vec q_1,\sigma}c_{\vec k_2,\sigma}c^\dagger_{\vec k_3,\sigma^\prime}c_{\vec k_4,\sigma^\prime} + \text{h.c.}\right]}_{T_2^\dagger + T_2}\\
		   &\underbrace{- \frac{1}{2}\sum_{\vec q_1,\vec k_2, \vec q_2, \vec k_4,\sigma,\sigma^\prime}W_{\vec q_1,\vec k_2, \vec q_2, \vec k_4} \sigma\sigma^\prime \left[c^\dagger_{\vec q_1,\sigma}c_{\vec k_2,\sigma}c^\dagger_{\vec q_2,\sigma^\prime}c_{\vec k_4,\sigma^\prime} + \text{h.c.}\right]}_{T_3^\dagger + T_3}\\
		   &\underbrace{- \frac{1}{2}\sum_{\vec q_1,\vec k_2,\vec k_3,\vec q_2,\sigma,\sigma^\prime}W_{\vec q_1,\vec k_2,\vec k_3,\vec q_2} \sigma\sigma^\prime c^\dagger_{\vec q_1,\sigma}c_{\vec k_2,\sigma}c^\dagger_{\vec k_3,\sigma^\prime}c_{\vec q_2,\sigma^\prime}}_{T_4}\\
		   &\underbrace{- \frac{1}{2}\sum_{\vec q_1,\vec k_2,\vec k_3,\vec q_2,\sigma,\sigma^\prime}W_{\vec q_1,\vec k_2,\vec k_3,\vec q_2} \sigma\sigma^\prime c^\dagger_{\vec k_3,\sigma^\prime}c_{\vec q_2,\sigma^\prime}c^\dagger_{\vec q_1,\sigma}c_{\vec k_2,\sigma}}_{T_5}\\
		   &\underbrace{- \frac{1}{2}\sum_{\vec q_1,\vec k_2, \vec q_2, \vec q_3,\sigma,\sigma^\prime}W_{\vec q_1,\vec k_2, \vec q_2, \vec q_3} \sigma\sigma^\prime \left[c^\dagger_{\vec q_1,\sigma}c_{\vec k_2,\sigma}c^\dagger_{\vec q_2,\sigma^\prime}c_{\vec q_3,\sigma^\prime} + \text{h.c.}\right]}_{T_6 + T_6^\dagger}~.
\end{aligned}\end{equation}
The first term \(T_1\) is an impurity-mediated scattering between the states \(\vec q\) at energy \(\varepsilon_j\) and the states \(\vec k\) at energies below \(\varepsilon_j\). Terms \(T_2\) through \(T_6\) involve two-particle scattering between these momentum states involving an increasing number of participating states from the isoenergetic shell \(\varepsilon_j\), through the Hubbard-like local term \(W\). The renormalisation of the Hamiltonian is constructed from the general expression
\begin{equation}\begin{aligned}
	\Delta H^{(j)} = H_X \frac{1}{\omega- H_D} H_X~.
\end{aligned}\end{equation}

\section{Renormalisation of the bath correlation term \(W\)}
In order to lead to a renormalisation of the \(W-\)term, there must be a total of four uncontracted momentum indices \(k_i\) and two contracted indices \(q_1, q_2\). The following combinations of scattering processes are compatible: (i) \(T_2^\dagger G T_6 + T_6^\dagger G T_2\), (ii) \(T_3^\dagger G T_3 + T_3 G T_3^\dagger\), (iii) \(T_4 G T_4\), (iv) \(T_5 G T_5\) and (v) \(T_4 G T_5 + T_5 G T_4\).

The first term is of the form
\begin{equation}\begin{aligned}
	T_2^\dagger G T_6 = \sum \sigma_1\sigma_1^\prime W_{\vec q_1,\vec k_2,\vec k_3,\vec k_4} c^\dagger_{\vec q_1, \sigma_1}c_{\vec k_2,\sigma_1}c^\dagger_{\vec k_3,\sigma_1^\prime}c_{\vec k_4,\sigma_1^\prime} \frac{1}{\omega - H_D}\sigma_1\sigma_2^\prime W_{\vec q_1,\vec q_2,\vec q_2,\vec k_2^\prime} \hat n_{\vec q_2,\sigma_2^\prime}c^\dagger_{\vec k_2^\prime,\sigma_1}c_{\vec q_1,\sigma_1}~.
\end{aligned}\end{equation}
The change in occupancy of the state \(\vec q_1\sigma_1\) from 1 to 0 leads to an excited state energy \(H_D = \varepsilon(q_1)\tau_{q_1\sigma_1} - \frac{1}{2}W_{\vec q_1}\left(\hat n_{\vec q_1, \uparrow} - \hat n_{\vec q_1, \uparrow}\right)^2 = -\frac{1}{2}\varepsilon(q_1) - \frac{1}{2}W_{\vec q_1}\), where \(W_{\vec q_1}\) is a shorthand for \(W_{\vec q_1,\vec q_1,\vec q_1,\vec q_1}\). The number operator \(\hat n_{\vec q_2,\sigma_2^\prime}\) projects the initial state to that in which \(\vec q_2\) is occupied. The operator \(c_{\vec q_1,\sigma_1}\) can be combined with its hermitian conjugate on the other side to give another number operator, leading to another projection. Both the number operators reduce to unity upon acting on the projected states:
\begin{equation}\begin{aligned}
	T_2^\dagger G T_6 = \sum \sigma_1^2\sigma_1^\prime W_{\vec q_1,\vec k_2,\vec k_3,\vec k_4} W_{\vec q_1,\vec q_2,\vec q_2,\vec k_2^\prime} c_{\vec k_2,\sigma_1}c^\dagger_{\vec k_3,\sigma_1^\prime}c_{\vec k_4,\sigma_1^\prime} c^\dagger_{\vec k_2^\prime,\sigma_1}\frac{1}{\omega + \frac{1}{2}\varepsilon(q_1) + \frac{1}{2}W_{\vec q_1}}\sum_{\sigma_2^\prime}\sigma_2^\prime  ~.
\end{aligned}\end{equation}
Since there is a sum over the spin factor \(\sigma_2^\prime = \pm 1\), this term vanishes identically. The particle-hole transformed term \(T_6^\dagger G T_2\) vanishes for the same reason.

We now consider the second term \(T_3^\dagger G T_3\):
\begin{equation}\begin{aligned}
	T_3^\dagger G T_3 &= \frac{1}{4}\sum W_{\vec q_1,\vec k_2, \vec q_2, \vec k_4} \sigma\sigma^\prime c^\dagger_{\vec q_1,\sigma}c_{\vec k_2,\sigma}c^\dagger_{\vec q_2,\sigma^\prime}c_{\vec k_4,\sigma^\prime} \frac{1}{\omega - H_D} W_{\vec q_1,\vec k_2^\prime, \vec q_2, \vec k_4^\prime} \sigma\sigma^\prime c^\dagger_{\vec k^\prime_4,\sigma^\prime} c_{\vec q_2,\sigma^\prime} c^\dagger_{\vec k^\prime_2,\sigma} c_{\vec q_1,\sigma}\\
			  &= \frac{1}{4}\sum  \hat n_{\vec q_1,\sigma} \hat n_{\vec q_2,\sigma^\prime} c_{\vec k_2,\sigma} c_{\vec k_4,\sigma^\prime} c^\dagger_{\vec k^\prime_4,\sigma^\prime} c^\dagger_{\vec k^\prime_2,\sigma} \frac{W_{\vec q_1,\vec k_2, \vec q_2, \vec k_4} W_{\vec q_1,\vec k_2^\prime, \vec q_2, \vec k_4^\prime}}{\omega + \frac{1}{2}\left[\varepsilon(q_1) + \varepsilon(q_2)\right] + \frac{1}{2}\left(W_{\vec q_1} + W_{\vec q_2}\right)}\\
			  &= \frac{1}{4}\sum_{4^\prime,4,2^\prime,2,\atop{\sigma,\sigma^\prime}} c^\dagger_{\vec k^\prime_4,\sigma^\prime} c_{\vec k_4,\sigma^\prime} c^\dagger_{\vec k^\prime_2,\sigma}c_{\vec k_2,\sigma}  \sum_{\vec q_1, \vec q_2 \in \text{PS}}\frac{W_{\vec q_1,\vec k_2, \vec q_2, \vec k_4} W_{\vec q_1,\vec k_2^\prime, \vec q_2, \vec k_4^\prime}}{\omega + \frac{1}{2}\left[\varepsilon(q_1) + \varepsilon(q_2)\right] + \frac{1}{2}\left(W_{\vec q_1} + W_{\vec q_2}\right)}~,
\end{aligned}\end{equation}
where \(\vec q_1, \vec q_2\) are summed over all momentum states in the isoenergetic shell and in the particle sector (PS) (states are occupied in the ground state).

The particle-hole transformed term is \(T_3 G T_3^\dagger\):
\begin{equation}\begin{aligned}
	T_3 G T_3^\dagger &= \frac{1}{4}\sum  W_{\vec q_1,\vec k_2^\prime, \vec q_2, \vec k_4^\prime} \sigma\sigma^\prime c^\dagger_{\vec k^\prime_4,\sigma^\prime} c_{\vec q_2,\sigma^\prime} c^\dagger_{\vec k^\prime_2,\sigma} c_{\vec q_1,\sigma}\frac{1}{\omega - H_D}W_{\vec q_1,\vec k_2, \vec q_2, \vec k_4} \sigma\sigma^\prime c^\dagger_{\vec q_1,\sigma}c_{\vec k_2,\sigma}c^\dagger_{\vec q_2,\sigma^\prime}c_{\vec k_4,\sigma^\prime} \\
			  &= \frac{1}{4}\sum \left(1 - \hat n_{\vec q_1,\sigma}\right) \left(1 - \hat n_{\vec q_2,\sigma^\prime}\right) c^\dagger_{\vec k^\prime_4,\sigma^\prime} c^\dagger_{\vec k^\prime_2,\sigma}  c_{\vec k_2,\sigma} c_{\vec k_4,\sigma^\prime}\frac{W_{\vec q_1,\vec k_2, \vec q_2, \vec k_4} W_{\vec q_1,\vec k_2^\prime, \vec q_2, \vec k_4^\prime}}{\omega - \frac{1}{2}\left[\varepsilon(q_1) + \varepsilon(q_2)\right] + \frac{1}{2}\left(W_{\vec q_1} + W_{\vec q_2}\right)}\\
			  &= \frac{1}{4}\sum_{4^\prime,4,2^\prime,2,\atop{\sigma,\sigma^\prime}} c^\dagger_{\vec k^\prime_4,\sigma^\prime}  c_{\vec k_4,\sigma^\prime} c^\dagger_{\vec k^\prime_2,\sigma}  c_{\vec k_2,\sigma} \sum_{\vec q_1, \vec q_2 \in \text{HS}}\frac{W_{\vec q_1,\vec k_2, \vec q_2, \vec k_4} W_{\vec q_1,\vec k_2^\prime, \vec q_2, \vec k_4^\prime}}{\omega - \frac{1}{2}\left[\varepsilon(q_1) + \varepsilon(q_2)\right] + \frac{1}{2}\left(W_{\vec q_1} + W_{\vec q_2}\right)}~,
\end{aligned}\end{equation}
where the hole projectors \(\left(1 - \hat n_{\vec q_1,\sigma}\right) \left(1 - \hat n_{\vec q_2,\sigma^\prime}\right)\) force the momenta \(\vec q_1,\vec q_2\) to extend over all states in the hole sector (states that are unoccupied in the ground state). The change in the sign of \(\left[\varepsilon(q_1) + \varepsilon(q_2)\right]\) in the denominator compared to the denominator in \(T_3^\dagger G T_3\) is for the same reason.

The remaining sets of terms that we need to consider are:
\begin{equation}\begin{aligned}
	T_4 G T_4 &= \frac{1}{4}\sum  W_{\vec q_1,\vec k_1, \vec q_2, \vec k_2} \sigma\sigma^\prime c^\dagger_{\vec q_1,\sigma}c_{\vec k_2,\sigma}c^\dagger_{\vec k_1,\sigma^\prime}c_{\vec q_2,\sigma^\prime} \frac{1}{\omega - H_D}W_{\vec q_1, \vec k_3, \vec q_2, \vec k_4} \sigma\sigma^\prime c^\dagger_{\vec q_2,\sigma^\prime}c_{\vec k_4,\sigma^\prime}c^\dagger_{\vec k_3,\sigma}c_{\vec q_1,\sigma}\\
		  &= \frac{1}{4}\sum \hat n_{\vec q_1,\sigma} \left(1 - \hat n_{\vec q_2,\sigma^\prime}\right) c_{\vec k_2,\sigma}c^\dagger_{\vec k_1,\sigma^\prime}c_{\vec k_4,\sigma^\prime}c^\dagger_{\vec k_3,\sigma} \frac{W_{\vec q_1,\vec k_1, \vec q_2, \vec k_2} W_{\vec q_1, \vec k_3, \vec q_2, \vec k_4}}{\omega + \frac{1}{2}\varepsilon\left(q_1\right) - \frac{1}{2}\varepsilon\left(q_2\right) + \frac{1}{2}\left(W_{\vec q_1} + W_{\vec q_2}\right)}  \\
		  &= -\frac{1}{4}\sum_{1,2,3,4,\atop{\sigma,\sigma^\prime}} c^\dagger_{\vec k_1,\sigma^\prime}c_{\vec k_4,\sigma^\prime}c^\dagger_{\vec k_3,\sigma}c_{\vec k_2,\sigma} \sum_{\vec q_1 \in \text{PS}, \atop{\vec q_2 \in \text{HS}}}\frac{W_{\vec q_1,\vec k_1, \vec q_2, \vec k_2} W_{\vec q_1, \vec k_3, \vec q_2, \vec k_4}}{\omega + \frac{1}{2}\varepsilon\left(q_1\right) - \frac{1}{2}\varepsilon\left(q_2\right) + \frac{1}{2}\left(W_{\vec q_1} + W_{\vec q_2}\right)}~.
\end{aligned}\end{equation}

\begin{equation}\begin{aligned}
	T_5 G T_5 &= \frac{1}{4}\sum  W_{\vec q_1,\vec k_1, \vec q_2, \vec k_2} \sigma\sigma^\prime c^\dagger_{\vec k_1,\sigma^\prime}c_{\vec q_2,\sigma^\prime}c^\dagger_{\vec q_1,\sigma}c_{\vec k_2,\sigma} \frac{1}{\omega - H_D}W_{\vec q_1, \vec k_3, \vec q_2, \vec k_4} \sigma\sigma^\prime c^\dagger_{\vec k_3,\sigma}c_{\vec q_1,\sigma}c^\dagger_{\vec q_2,\sigma^\prime}c_{\vec k_4,\sigma^\prime}\\
		  &= \frac{1}{4}\sum \hat n_{\vec q_1,\sigma} \left(1 - \hat n_{\vec q_2,\sigma^\prime}\right) c^\dagger_{\vec k_1,\sigma^\prime}c_{\vec k_2,\sigma}c^\dagger_{\vec k_3,\sigma}c_{\vec k_4,\sigma^\prime} \frac{W_{\vec q_1,\vec k_1, \vec q_2, \vec k_2} W_{\vec q_1, \vec k_3, \vec q_2, \vec k_4}}{\omega + \frac{1}{2}\varepsilon\left(q_1\right) - \frac{1}{2}\varepsilon\left(q_2\right) + \frac{1}{2}\left(W_{\vec q_1} + W_{\vec q_2}\right)}  \\
		  &= -\frac{1}{4}\sum_{1,2,3,4,\atop{\sigma,\sigma^\prime}} c^\dagger_{\vec k_1,\sigma^\prime} c_{\vec k_4,\sigma^\prime} c^\dagger_{\vec k_3,\sigma} c_{\vec k_2,\sigma} \sum_{\vec q_1 \in \text{PS}, \atop{\vec q_2 \in \text{HS}}}\frac{W_{\vec q_1,\vec k_1, \vec q_2, \vec k_2} W_{\vec q_1, \vec k_3, \vec q_2, \vec k_4}}{\omega + \frac{1}{2}\varepsilon\left(q_1\right) - \frac{1}{2}\varepsilon\left(q_2\right) + \frac{1}{2}\left(W_{\vec q_1} + W_{\vec q_2}\right)}~.
\end{aligned}\end{equation}

\begin{equation}\begin{aligned}
	T_4 G T_5 &= \frac{1}{4}\sum  W_{\vec q_1,\vec k_1, \vec q_2, \vec k_2} \sigma\sigma^\prime c^\dagger_{\vec q_1,\sigma}c_{\vec k_2,\sigma}c^\dagger_{\vec k_1,\sigma^\prime}c_{\vec q_2,\sigma^\prime} \frac{1}{\omega - H_D}W_{\vec q_1, \vec k_3, \vec q_2, \vec k_4} \sigma\sigma^\prime c^\dagger_{\vec k_3,\sigma}c_{\vec q_1,\sigma}c^\dagger_{\vec q_2,\sigma^\prime}c_{\vec k_4,\sigma^\prime}\\
		  &= -\frac{1}{4}\sum \hat n_{\vec q_1,\sigma} \left(1 - \hat n_{\vec q_2,\sigma^\prime}\right) c_{\vec k_2,\sigma} c^\dagger_{\vec k_1,\sigma^\prime}c^\dagger_{\vec k_3,\sigma}c_{\vec k_4,\sigma^\prime} \frac{W_{\vec q_1,\vec k_1, \vec q_2, \vec k_2} W_{\vec q_1, \vec k_3, \vec q_2, \vec k_4}}{\omega + \frac{1}{2}\varepsilon\left(q_1\right) - \frac{1}{2}\varepsilon\left(q_2\right) + \frac{1}{2}\left(W_{\vec q_1} + W_{\vec q_2}\right)}  \\
		  &= -\frac{1}{4}\sum_{1,2,3,4,\atop{\sigma,\sigma^\prime}} c^\dagger_{\vec k_1,\sigma^\prime}c_{\vec k_4,\sigma^\prime}c^\dagger_{\vec k_3,\sigma} c_{\vec k_2,\sigma} \sum_{\vec q_1 \in \text{PS}, \atop{\vec q_2 \in \text{HS}}}\frac{W_{\vec q_1,\vec k_1, \vec q_2, \vec k_2} W_{\vec q_1, \vec k_3, \vec q_2, \vec k_4}}{\omega + \frac{1}{2}\varepsilon\left(q_1\right) - \frac{1}{2}\varepsilon\left(q_2\right) + \frac{1}{2}\left(W_{\vec q_1} + W_{\vec q_2}\right)}~.
\end{aligned}\end{equation}

\begin{equation}\begin{aligned}
	T_5 G T_4 &= \frac{1}{4}\sum W_{\vec q_1, \vec k_3, \vec q_2, \vec k_4} \sigma\sigma^\prime c^\dagger_{\vec k_3,\sigma}c_{\vec q_2,\sigma}c^\dagger_{\vec q_1,\sigma^\prime}c_{\vec k_4,\sigma^\prime} \frac{1}{\omega - H_D} W_{\vec q_1,\vec k_1, \vec q_2, \vec k_2} \sigma\sigma^\prime c^\dagger_{\vec q_2,\sigma}c_{\vec k_2,\sigma}c^\dagger_{\vec k_1,\sigma^\prime}c_{\vec q_1,\sigma^\prime}\\
		  &= -\frac{1}{4}\sum \hat n_{\vec q_1,\sigma} \left(1 - \hat n_{\vec q_2,\sigma^\prime}\right) c^\dagger_{\vec k_3,\sigma}c_{\vec k_4,\sigma^\prime} c_{\vec k_2,\sigma} c^\dagger_{\vec k_1,\sigma^\prime} \frac{W_{\vec q_1,\vec k_1, \vec q_2, \vec k_2} W_{\vec q_1, \vec k_3, \vec q_2, \vec k_4}}{\omega + \frac{1}{2}\varepsilon\left(q_1\right) - \frac{1}{2}\varepsilon\left(q_2\right) + \frac{1}{2}\left(W_{\vec q_1} + W_{\vec q_2}\right)}  \\
		  &= -\frac{1}{4}\sum_{1,2,3,4,\atop{\sigma,\sigma^\prime}} c^\dagger_{\vec k_3,\sigma} c_{\vec k_2,\sigma} c^\dagger_{\vec k_1,\sigma^\prime}c_{\vec k_4,\sigma^\prime}\sum_{\vec q_1 \in \text{PS}, \atop{\vec q_2 \in \text{HS}}}\frac{W_{\vec q_1,\vec k_1, \vec q_2, \vec k_2} W_{\vec q_1, \vec k_3, \vec q_2, \vec k_4}}{\omega + \frac{1}{2}\varepsilon\left(q_1\right) - \frac{1}{2}\varepsilon\left(q_2\right) + \frac{1}{2}\left(W_{\vec q_1} + W_{\vec q_2}\right)}~.
\end{aligned}\end{equation}

\end{document}
