\documentclass{revtex4-2}
\usepackage{amsmath,hyperref}
\begin{document}
\title{URG analysis of the extended Kondo model with interactions in various angular momentum channels}
\author{Abhirup Mukherjee, Siddhartha Lal}
\maketitle
\section{Hamiltonian}
We consider an impurity spin \(\vec S_d\) interacting with a two-dimensional tight-binding conduction bath through a general interaction of the form
\begin{equation}\begin{aligned}
		\sum_{{\bf k}_1, {\bf k}_2, \sigma_1, \sigma_2} J_{{\bf k}_1, {\bf k}_2} \vec{S}_d\cdot\frac{1}{2}\vec{\tau}_{\sigma_1,\sigma_2} c^\dagger_{{\bf k}_1,\sigma_1}c_{{\bf k}_2,\sigma_2}
\end{aligned}\end{equation}
where \(\vec \tau\) is the vector of sigma matrices and \({\bf k}_1,{\bf k}_2\) are momentum states of the conduction bath. The precise form of \(J_{{\bf k}_1, {\bf k}_2}\) depends on the symmetry of the symmetry of the impurity-bath interaction. We consider the following three cases:

(i)~a d-wave interaction, where the impurity couples with a coherent d-wave combination \(f_\sigma\) of the bath sites closest to it: \(f_\sigma \equiv \frac{1}{2}\left(c^\dagger_{L,\sigma} + c^\dagger_{R,\sigma} - c^\dagger_{U\sigma} - c^\dagger_{D\sigma}\right)\), where \(L,R,U\) and \(D\) indicate electrons at the positions \((x,y)=\left(-a,0\right), \left( a,0 \right) , \left( 0,a \right) \) and \(\left( 0,-a \right) \) respectively, \(a\) being the lattice spacing of the conduction bath lattice. The corresponding interaction (in real space) is of the form \(\vec{S}_d\cdot\sum_{\sigma_1,\sigma_2}\frac{1}{2}\vec{\tau}_{\sigma_1,\sigma_2}f^\dagger_{\sigma_1}f_{\sigma_2}\). When fourier-transformed to momentum space, it gives rise to the momentum-dependent Kondo coupling 
\begin{equation}\begin{aligned}\label{J_dwave}
	J_{{\bf k}_1, {\bf k}_2} = J\prod_{i=1,2}\left[\cos\left( ak_i^x \right) - \cos\left( ak_i^y \right) \right]~.
\end{aligned}\end{equation}
For reference, we define the Fourier transforms as \(c^\dagger_{L(R),\sigma} = \sum_{{\bf k}}c^\dagger_{{\bf k},\sigma}e^{-(+)ik^x a}\), \(c^\dagger_{U(D),\sigma} = \sum_{{\bf k}}c^\dagger_{{\bf k},\sigma}e^{-(+)ik^y a}\).

(ii)~a p-wave interaction, where the impurity couples with the p-wave electron \(p_\sigma \equiv \frac{1}{2}\left(c^\dagger_{L,\sigma} + c^\dagger_{R,\sigma} + c^\dagger_{U\sigma} + c^\dagger_{D\sigma}\right)\). The momentum-dependent Kondo coupling in this case is 
\begin{equation}\begin{aligned}\label{J_pwave}
	J_{{\bf k}_1, {\bf k}_2} = J\prod_{i=1,2}\left[\cos\left( ak_i^x \right) + \cos\left( ak_i^y \right) \right]
\end{aligned}\end{equation}
Since we are considering a 2-dimensional conduction bath in the tight-binding limit, such a Kondo coupling vanishes close to the Fermi surface. As a result, this case is not interesting for our purpose.

(iii)~a p-wave interaction, but without the off-site terms in the bath. This amounts to considering the Kondo interaction term \(J\vec{S}_d\cdot\sum_{\sigma_1,\sigma_2}\frac{1}{2}\vec{\tau}_{\sigma_1,\sigma_2}\frac{1}{4}\sum_{i=L,R,U,D}c^\dagger_{i,\sigma_1}c_{i,\sigma_2}\). The momentum-dependent Kondo coupling in this case is 
\begin{equation}\begin{aligned}\label{J_pwave_nooffsite}
	J_{{\bf k}_1, {\bf k}_2} = J_{{\bf k}_1 - {\bf k}_2} \equiv \frac{1}{2}J \cos\left[ a\left( k_1^x - k_2^x \right) + x\to y \right]
\end{aligned}\end{equation}
This does not vanish identically on the Fermi surface, and is therefore of potential interest to us.

We also consider correlation terms on the bath sites \(L,R,U\) and \(D\). In momentum space, it is of the general form
\begin{equation}\begin{aligned}
	-\frac{1}{2}\left[\sum_{1,2,\sigma}W_{1,2}~ c^\dagger_{{\bf k}_1,\sigma}c_{{\bf k}_2,\sigma} - \sum_{1, 2, 3, 4,\sigma}W_{1,2,3,4}~ c^\dagger_{{\bf k}_1,\sigma}c_{{\bf k}_2,\sigma}c^\dagger_{{\bf k}_3,\bar\sigma}c_{{\bf k}_4,\bar\sigma}\right] ~,
\end{aligned}\end{equation}
where the form of \(W_{\left\{{\bf k}_i\right\}}\) is again determined by the orbitals participating in the interaction. We ignore the p-wave interaction (for the same reason as above) and consider just the d-wave interaction and a p-wave interaction without off-site terms. These two cases lead to the following Hamiltonian structures:
(i)~a d-wave interaction, of the form \(-\frac{W}{2}\left(f^\dagger_{\uparrow}f_{\uparrow} - f^\dagger_{\downarrow}f_{\downarrow}\right)^2\) in real space, leading to the momentum-dependence
\begin{equation}\begin{aligned}\label{W_dwave}
	W_{1,2,3,4} = W\prod_{i=1,2,3,4}\left[\cos\left( ak_i^x \right) - \cos\left( ak_i^y \right) \right]~.
\end{aligned}\end{equation}
(ii)~a p-wave interaction without off-site terms, \(-\frac{1}{4}\frac{W}{2}\sum_{i=L,R,U,D}\left(c^\dagger_{i,\uparrow}c_{i,\uparrow} - c^\dagger_{i,\downarrow}c_{i,\downarrow}\right)^2\), leading to the following form in momentum space:
\begin{equation}\begin{aligned}\label{W_pwave_nooffsite}
	W_{\left\{{\bf k}_i\right\}} = W_{{\bf k}_1 - {\bf k}_2, {\bf k}_3 - {\bf k}_4} = \frac{1}{2}W\left[\cos\left(a\left(k_1^x - k_2^x + k_3^x - k_4^x\right)\right) + x \to y\right]~.
\end{aligned}\end{equation}

To summarise, the total Hamiltonian we consider is of the form
\begin{equation}\begin{aligned}\label{hamiltonian}
	H = -2t\sum_{kx,k y}\left[\cos(ak_x) + \cos(ak_y)\right] c^\dagger_{{\bf k},\sigma}c_{{\bf k},\sigma} + \sum_{{\bf k}_1, {\bf k}_2, \sigma_1, \sigma_2} J_{{\bf k}_1, {\bf k}_2} \vec{S}_d\cdot\frac{1}{2}\vec{\tau}_{\sigma_1,\sigma_2} c^\dagger_{{\bf k}_1,\sigma_1}c_{{\bf k}_2,\sigma_2} \\
-\frac{1}{2}\left[\sum_{1,2,\sigma}W_{1,2}~ c^\dagger_{{\bf k}_1,\sigma}c_{{\bf k}_2,\sigma} - \sum_{1, 2, 3, 4,\sigma}W_{1,2,3,4}~ c^\dagger_{{\bf k}_1,\sigma}c_{{\bf k}_2,\sigma}c^\dagger_{{\bf k}_3,\bar\sigma}c_{{\bf k}_4,\bar\sigma}\right]~,
\end{aligned}\end{equation}
where \(J_{\left\{ {\bf k}_i \right\} }\) takes one of the two forms in eqs.~\ref{J_dwave} and \ref{J_pwave_nooffsite}, and \(W_{\left\{{\bf k}_i\right\}}\) can similarly take either of the two forms in eqs.~\ref{W_dwave} and \ref{W_pwave_nooffsite}. 

\section{RG scheme}
At any given step \(j\) of the RG procedure, we decouple the states \(\left\{ {\bf q} \right\} \) on the isoenergetic surface of energy \(\varepsilon_j\). The diagonal Hamiltonian \(H_D\) for this step consists of all terms that do not change the occupancy of the states \(\left\{{\bf q}\right\}\):
\begin{equation}\begin{aligned}
	H_D^{(j)} = \varepsilon_j\sum_{q,\sigma}\tau_{q,\sigma} + \frac{1}{2}\sum_{{\bf q}}J_{{\bf q}, {\bf q}}S_d^z\left(\hat n_{{\bf q}, \uparrow} - \hat n_{{\bf q}, \downarrow}\right) - \frac{1}{2}\sum_{{\bf q}}\left[W^S_{{\bf q}}\left(\hat n_{{\bf q}, \uparrow} + \hat n_{{\bf q}, \downarrow}\right) - 2W^A_{{\bf q}}\hat n_{{\bf q}, \uparrow}\hat n_{{\bf q}, \downarrow}\right]~.
\end{aligned}\end{equation}
where \(\tau = \hat n - 1/2\) and \(W_{{\bf q}}\) is a shorthand for \(W_{{\bf q},{\bf q},{\bf q},{\bf q}}\). The three terms, respectively, are the kinetic energy of the momentum states on the isoenergetic shell that we are decoupling, the Ising interaction energy between the impurity spin and the spins formed by these momentum states and, finally, the local correlation energy associated with these states arising from the \(W\) term.

The off-diagonal part of the Hamiltonian on the other hand leads to scattering in the states \(\left\{ {\bf q} \right\} \). We now list these terms, classified by the coupling they originate from.
\par\noindent
{\bf Arising from \(J\):}
\begin{equation}\begin{aligned}
	T_1^\dagger + T_1 = \sum_{{\bf k}, {\bf q}, \sigma,\sigma^\prime}J_{{\bf k}, {\bf q}} \vec{S}_d\cdot\vec{\tau}_{\sigma,\sigma^\prime}\left[c^\dagger_{{\bf q}\sigma}c_{{\bf k},\sigma^\prime} + \text{h.c.}\right],\quad T_7 = \sum_{{\bf q}, {\bf q}^\prime, \sigma,\sigma^\prime}J_{{\bf k}, {\bf q}} \vec{S}_d\cdot\vec{\tau}_{\sigma,\sigma^\prime}c^\dagger_{{\bf q}\sigma}c_{{\bf q}^\prime,\sigma^\prime}
\end{aligned}\end{equation}
\par\noindent
{\bf Arising from potential scattering terms within \(W\):}
\begin{equation}\begin{aligned}
	T_{S2}^\dagger + T_{S2} &= - \frac{1}{2}\sum_{{\bf q} \in \varepsilon_j}\sum_{{\bf k}_1 < \varepsilon_j}\sum_{\sigma} \left[W_{{\bf q},{\bf k}_1}\left(c^\dagger_{{\bf q},\sigma}c_{{\bf k}_1,\sigma} + \text{h.c.}\right]\\
				&= - \frac{1}{2}\sum_{{\bf q} \in \varepsilon_j}\sum_{{\bf k}_2, {\bf k}_3, {\bf k}_4 < \varepsilon_j}\sum_{\sigma} \left[W^S_{{\bf q},{\bf k}_2,{\bf k}_3,{\bf k}_4}\left(c^\dagger_{{\bf q},\sigma}c_{{\bf k}_2,\sigma}c^\dagger_{{\bf k}_3,\sigma}c_{{\bf k}_4,\sigma} + \left(\delta_{{\bf k}_3,{\bf k}_2} - c_{{\bf k}_2,\sigma}c^\dagger_{{\bf k}_3,\sigma}\right)c^\dagger_{{\bf q},\sigma}c_{{\bf k}_4,\sigma}\right) + \text{h.c.}\right]\\
				&= - \frac{1}{2}\sum_{{\bf q} \in \varepsilon_j}\sum_{{\bf k}_4 < \varepsilon_j}\sum_{\sigma} \left[W^S_{{\bf q},{\bf k}_4}c^\dagger_{{\bf q},\sigma}c_{{\bf k}_4,\sigma} + \text{h.c.}\right]
\end{aligned}\end{equation}
\begin{equation}\begin{aligned}
	T_{S3}^\dagger + T_{S3} &= - \frac{1}{2}\sum_{{\bf q}, {\bf q}^\prime \in \varepsilon_j}\sum_{{\bf k}_2, {\bf k}_3 < \varepsilon_j}\sum_{\sigma} \left[W^S_{{\bf q},{\bf k}_2, {\bf q}^\prime, {\bf k}_3}c^\dagger_{{\bf q},\sigma}c_{{\bf k}_2,\sigma}c^\dagger_{{\bf q}^\prime,\sigma}c_{{\bf k}_3,\sigma} + \text{h.c.}\right]\\
\end{aligned}\end{equation}
\begin{equation}\begin{aligned}
	T_{S4} &= - \frac{1}{2}\sum_{{\bf q}, {\bf q}^\prime \in \varepsilon_j}\sum_{{\bf k}_2, {\bf k}_3 < \varepsilon_j}\sum_{\sigma}W^S_{{\bf q},{\bf q}^\prime,{\bf k}_2,{\bf k}_3} \left(c^\dagger_{{\bf q},\sigma}c_{{\bf q}^\prime,\sigma}c^\dagger_{{\bf k}_2,\sigma}c_{{\bf k}_3,\sigma} + c^\dagger_{{\bf k}_2,\sigma}c_{{\bf k}_3,\sigma}c^\dagger_{{\bf q},\sigma}c_{{\bf q}^\prime,\sigma}\right) \\
				&= - \frac{1}{2}\sum_{{\bf q}, {\bf q}^\prime \in \varepsilon_j}\sum_{{\bf k}_2, {\bf k}_3 < \varepsilon_j}\sum_{\sigma}2W^S_{{\bf q},{\bf q}^\prime,{\bf k}_2,{\bf k}_3} c^\dagger_{{\bf q},\sigma}c_{{\bf q}^\prime,\sigma}c^\dagger_{{\bf k}_2,\sigma}c_{{\bf k}_3,\sigma}\\
\end{aligned}\end{equation}
\begin{equation}\begin{aligned}
	T_{S5} &= - \frac{1}{2}\sum_{{\bf q}, {\bf q}^\prime \in \varepsilon_j}\sum_{{\bf k}_2, {\bf k}_3 < \varepsilon_j}\sum_{\sigma}W^S_{{\bf q},{\bf k}_2,{\bf k}_3,{\bf q}^\prime} \left(c^\dagger_{{\bf q},\sigma}c_{{\bf k}_2,\sigma}c^\dagger_{{\bf k}_3,\sigma}c_{{\bf q}^\prime,\sigma} + c^\dagger_{{\bf k}_3,\sigma}c_{{\bf q}^\prime,\sigma}c^\dagger_{{\bf q},\sigma}c_{{\bf k}_2,\sigma}\right)\\
	       &= - \frac{1}{2}\sum_{{\bf q}, {\bf q}^\prime \in \varepsilon_j}\sum_{{\bf k}_2, {\bf k}_3 < \varepsilon_j}\sum_{\sigma}2W^S_{{\bf q},{\bf k}_2,{\bf k}_3,{\bf q}^\prime} c^\dagger_{{\bf q},\sigma}c_{{\bf k}_2,\sigma}c^\dagger_{{\bf k}_3,\sigma}c_{{\bf q}^\prime,\sigma}\\
\end{aligned}\end{equation}
\begin{equation}\begin{aligned}
	T_{S6}^\dagger + T_{S6} &= - \frac{1}{2}\sum_{{\bf q}, {\bf q}^\prime, {\bf q}^{\prime\prime} \in \varepsilon_j}\sum_{{\bf k}_1 < \varepsilon_j}\sum_{\sigma} \left[W^S_{{\bf q},{\bf q}^\prime,{\bf q}^{\prime\prime}, {\bf k}_1}\left(c^\dagger_{{\bf q},\sigma}c_{{\bf q}^\prime,\sigma}c^\dagger_{{\bf q}^{\prime\prime},\sigma}c_{{\bf k}_1,\sigma} + c^\dagger_{{\bf q}^{\prime\prime},\sigma}c_{{\bf k}_1,\sigma}c^\dagger_{{\bf q},\sigma}c_{{\bf q}^\prime,\sigma}\right) + \text{h.c.}\right]\\
				&= - \frac{1}{2}\sum_{{\bf q}, {\bf q}^\prime, {\bf q}^{\prime\prime} \in \varepsilon_j}\sum_{{\bf k}_1 < \varepsilon_j}\sum_{\sigma} \left[W^S_{{\bf q},{\bf q}^\prime,{\bf q}^{\prime\prime}, {\bf k}_1}\left(c^\dagger_{{\bf q},\sigma}c_{{\bf q}^\prime,\sigma}c^\dagger_{{\bf q}^{\prime\prime},\sigma} - c^\dagger_{{\bf q},\sigma}\left(\delta_{{\bf q}^\prime,{\bf q}^{\prime\prime}} - c_{{\bf q}^\prime,\sigma}c^\dagger_{{\bf q}^{\prime\prime},\sigma}\right)\right) c_{{\bf k}_1,\sigma} + \text{h.c.}\right]\\
				&= - \frac{1}{2}\sum_{{\bf q}, {\bf q}^\prime, {\bf q}^{\prime\prime} \in \varepsilon_j}\sum_{{\bf k}_1 < \varepsilon_j}\sum_{\sigma}\left[2W^S_{{\bf q},{\bf q}^\prime,{\bf q}^{\prime\prime}, {\bf k}_1}c^\dagger_{{\bf q},\sigma}c_{{\bf q}^\prime,\sigma}c^\dagger_{{\bf q}^{\prime\prime},\sigma}c_{{\bf k}_1,\sigma} + \text{h.c.}\right] + \frac{1}{2}\sum_{{\bf q} \in \varepsilon_j}\sum_{{\bf k}_1 < \varepsilon_j}\sum_{\sigma}\left[W^S_{{\bf q}, {\bf k}_1}c^\dagger_{{\bf q},\sigma}c_{{\bf k}_1,\sigma} + \text{h.c.}\right]~.\\
\end{aligned}\end{equation}
The last term of \(T_{S6}^\dagger + T_{S6}\) cancels out \(T_{S2}^\dagger + T_{S2}\), so the final list of terms is slightly shorter:
\begin{equation}\begin{aligned}
	T_{S3}^\dagger + T_{S3} &= - \frac{1}{2}\sum_{{\bf q}, {\bf q}^\prime \in \varepsilon_j}\sum_{{\bf k}_2, {\bf k}_3 < \varepsilon_j}\sum_{\sigma} W^S_{{\bf q},{\bf k}_2, {\bf q}^\prime, {\bf k}_3}\left[c^\dagger_{{\bf q},\sigma}c_{{\bf k}_2,\sigma}c^\dagger_{{\bf q}^\prime,\sigma}c_{{\bf k}_3,\sigma} + \text{h.c.}\right]\\
\end{aligned}\end{equation}
\begin{equation}\begin{aligned}
	T_{S4} &= - \frac{1}{2}\sum_{{\bf q}, {\bf q}^\prime \in \varepsilon_j}\sum_{{\bf k}_2, {\bf k}_3 < \varepsilon_j}\sum_{\sigma}2W^S_{{\bf q},{\bf q}^\prime,{\bf k}_2,{\bf k}_3} c^\dagger_{{\bf q},\sigma}c_{{\bf q}^\prime,\sigma}c^\dagger_{{\bf k}_2,\sigma}c_{{\bf k}_3,\sigma}\\
\end{aligned}\end{equation}
\begin{equation}\begin{aligned}
	T_{S5} &= - \frac{1}{2}\sum_{{\bf q}, {\bf q}^\prime \in \varepsilon_j}\sum_{{\bf k}_2, {\bf k}_3 < \varepsilon_j}\sum_{\sigma}2W^S_{{\bf q},{\bf k}_2,{\bf k}_3,{\bf q}^\prime} c^\dagger_{{\bf q},\sigma}c_{{\bf k}_2,\sigma}c^\dagger_{{\bf k}_3,\sigma}c_{{\bf q}^\prime,\sigma}\\
\end{aligned}\end{equation}
\begin{equation}\begin{aligned}
	T_{S6}^\dagger + T_{S6} &= - \frac{1}{2}\sum_{{\bf q}, {\bf q}^\prime, {\bf q}^{\prime\prime} \in \varepsilon_j}\sum_{{\bf k}_1 < \varepsilon_j}\sum_{\sigma}\left[2W^S_{{\bf q},{\bf q}^\prime,{\bf q}^{\prime\prime}, {\bf k}_1}c^\dagger_{{\bf q},\sigma}c_{{\bf q}^\prime,\sigma}c^\dagger_{{\bf q}^{\prime\prime},\sigma}c_{{\bf k}_1,\sigma} + \text{h.c.}\right]~.\\
\end{aligned}\end{equation}

{\bf Arising from \(W^A\):}
\begin{equation}\begin{aligned}
	T_{A2}^\dagger + T_{A2} &= \frac{1}{2}\sum_{{\bf q} \in \varepsilon_j}\sum_{{\bf k}_2, {\bf k}_3, {\bf k}_4 < \varepsilon_j}\sum_{\sigma} \left[W^A_{{\bf q},{\bf k}_2,{\bf k}_3,{\bf k}_4}\left(c^\dagger_{{\bf q},\sigma}c_{{\bf k}_2,\sigma}c^\dagger_{{\bf k}_3,\bar\sigma}c_{{\bf k}_4,\bar\sigma} + c^\dagger_{{\bf k}_3,\sigma}c_{{\bf k}_2,\sigma}c^\dagger_{{\bf q},\bar\sigma}c_{{\bf k}_4,\bar\sigma}\right) + \text{h.c.}\right]\\
				&= \frac{1}{2}\sum_{{\bf q} \in \varepsilon_j}\sum_{{\bf k}_2, {\bf k}_3, {\bf k}_4 < \varepsilon_j}\sum_{\sigma} \left[2W^A_{{\bf q},{\bf k}_2,{\bf k}_3,{\bf k}_4}c^\dagger_{{\bf q},\sigma}c_{{\bf k}_2,\sigma}c^\dagger_{{\bf k}_3,\bar\sigma}c_{{\bf k}_4,\bar\sigma} + \text{h.c.}\right]\\
\end{aligned}\end{equation}
\begin{equation}\begin{aligned}
	T_{A3}^\dagger + T_{A3} &= \frac{1}{2}\sum_{{\bf q}, {\bf q}^\prime \in \varepsilon_j}\sum_{{\bf k}_2, {\bf k}_3 < \varepsilon_j}\sum_{\sigma} \left[W^A_{{\bf q},{\bf k}_2, {\bf q}^\prime, {\bf k}_3}c^\dagger_{{\bf q},\sigma}c_{{\bf k}_2,\sigma}c^\dagger_{{\bf q}^\prime,\bar\sigma}c_{{\bf k}_3,\bar\sigma} + \text{h.c.}\right]\\
\end{aligned}\end{equation}
\begin{equation}\begin{aligned}
	T_{A4} &= \frac{1}{2}\sum_{{\bf q}, {\bf q}^\prime \in \varepsilon_j}\sum_{{\bf k}_2, {\bf k}_3 < \varepsilon_j}\sum_{\sigma}W^A_{{\bf q},{\bf q}^\prime,{\bf k}_2,{\bf k}_3} \left(c^\dagger_{{\bf q},\sigma}c_{{\bf q}^\prime,\sigma}c^\dagger_{{\bf k}_2,\bar\sigma}c_{{\bf k}_3,\bar\sigma} + c^\dagger_{{\bf k}_2,\sigma}c_{{\bf k}_3,\sigma}c^\dagger_{{\bf q},\bar\sigma}c_{{\bf q}^\prime,\bar\sigma}\right)\\
				&= \frac{1}{2}\sum_{{\bf q}, {\bf q}^\prime \in \varepsilon_j}\sum_{{\bf k}_2, {\bf k}_3 < \varepsilon_j}\sum_{\sigma}2W^A_{{\bf q},{\bf q}^\prime,{\bf k}_2,{\bf k}_3} c^\dagger_{{\bf q},\sigma}c_{{\bf q}^\prime,\sigma}c^\dagger_{{\bf k}_2,\bar\sigma}c_{{\bf k}_3,\bar\sigma}\\
\end{aligned}\end{equation}
\begin{equation}\begin{aligned}
	T_{A5} &= \frac{1}{2}\sum_{{\bf q}, {\bf q}^\prime \in \varepsilon_j}\sum_{{\bf k}_2, {\bf k}_3 < \varepsilon_j}\sum_{\sigma}W^A_{{\bf q},{\bf k}_2,{\bf k}_3,{\bf q}^\prime} \left(c^\dagger_{{\bf q},\sigma}c_{{\bf k}_2,\sigma}c^\dagger_{{\bf k}_3,\bar\sigma}c_{{\bf q}^\prime,\bar\sigma} + c^\dagger_{{\bf k}_3,\sigma}c_{{\bf q}^\prime,\sigma}c^\dagger_{{\bf q},\bar\sigma}c_{{\bf k}_2,\bar\sigma}\right) \\
	       &= \frac{1}{2}\sum_{{\bf q}, {\bf q}^\prime \in \varepsilon_j}\sum_{{\bf k}_2, {\bf k}_3 < \varepsilon_j}\sum_{\sigma}2W^A_{{\bf q},{\bf k}_2,{\bf k}_3,{\bf q}^\prime} c^\dagger_{{\bf q},\sigma}c_{{\bf k}_2,\sigma}c^\dagger_{{\bf k}_3,\bar\sigma}c_{{\bf q}^\prime,\bar\sigma} \\
\end{aligned}\end{equation}
\begin{equation}\begin{aligned}
	T_{A6}^\dagger + T_{A6} &= \frac{1}{2}\sum_{{\bf q}, {\bf q}^\prime, {\bf q}^{\prime\prime} \in \varepsilon_j}\sum_{{\bf k}_1 < \varepsilon_j}\sum_{\sigma} \left[W^A_{{\bf q},{\bf q}^\prime,{\bf q}^{\prime\prime}, {\bf k}_1}\left(c^\dagger_{{\bf q},\sigma}c_{{\bf q}^\prime,\sigma}c^\dagger_{{\bf q}^{\prime\prime},\bar\sigma}c_{{\bf k}_1,\bar\sigma} + c^\dagger_{{\bf q}^{\prime\prime},\sigma}c_{{\bf k}_1,\sigma}c^\dagger_{{\bf q},\bar\sigma}c_{{\bf q}^\prime,\bar\sigma}\right) + \text{h.c.}\right]\\
				&= \frac{1}{2}\sum_{{\bf q}, {\bf q}^\prime, {\bf q}^{\prime\prime} \in \varepsilon_j}\sum_{{\bf k}_1 < \varepsilon_j}\sum_{\sigma} \left[2W^A_{{\bf q},{\bf q}^\prime,{\bf q}^{\prime\prime}, {\bf k}_1}c^\dagger_{{\bf q},\sigma}c_{{\bf q}^\prime,\sigma}c^\dagger_{{\bf q}^{\prime\prime},\bar\sigma}c_{{\bf k}_1,\bar\sigma} + \text{h.c.}\right]\\
\end{aligned}\end{equation}
The first term \(T_1\) is an impurity-mediated scattering between the states \({\bf q}\) at energy \(\varepsilon_j\) and the states \({\bf k}\) at energies below \(\varepsilon_j\). Terms \(T_2\) through \(T_6\) involve two-particle scattering between these momentum states involving an increasing number of participating states from the isoenergetic shell \(\varepsilon_j\), through the Hubbard-like local term \(W\). The renormalisation of the Hamiltonian is constructed from the general expression
\begin{equation}\begin{aligned}
	\Delta H^{(j)} = H_X \frac{1}{\omega- H_D} H_X~.
\end{aligned}\end{equation}

\section{Renormalisation of the bath correlation term \(W\)}
In order to lead to a renormalisation of the \(W-\)term, there must be a total of four uncontracted momentum indices \(k_i\) and two contracted indices \(q_1, q_2\). The following combinations of scattering processes are compatible: (i) \(T_2^\dagger G T_6 + T_6^\dagger G T_2\), (ii) \(T_3^\dagger G T_3 + T_3 G T_3^\dagger\), (iii) \(T_4 G T_4\), (iv) \(T_5 G T_5\) and (v) \(T_4 G T_5 + T_5 G T_4\).

\subsection{Correlated scattering involve one electron on the shell \(\varepsilon_j\)}
The first term is of the form
\begin{equation}\begin{aligned}
		T_2 G T_6^\dagger = \sum_{\left\{ {\bf k}_i \right\} }\sum_{\sigma,\sigma^\prime}\sum_{{\bf q}, {\bf q}^\prime} \sigma\sigma^\prime W_{{\bf k}_1,{\bf k}_2,{\bf k}_3,{\bf q}} c^\dagger_{{\bf k}_1, \sigma^\prime}c_{{\bf k}_2,\sigma^\prime}c^\dagger_{{\bf k}_3,\sigma}c_{{\bf q},\sigma} \frac{1}{\omega - H_D}\left(W_{{{\bf q}^\prime},{{\bf q}^\prime},{\bf q},{\bf k}_4} \hat n_{{{\bf q}^\prime},\bar\sigma}c^\dagger_{{\bf q},\sigma}c_{{\bf k}_4,\sigma} - W_{{{\bf q}^\prime},{{\bf q}^\prime},{\bf q},{\bf k}_4}\hat n_{{{\bf q}^\prime},\sigma}c^\dagger_{{\bf q},\sigma}c_{{\bf k}_4,\sigma} \right.\\
		\left.- W_{{\bf q},{{\bf q}^\prime},{{\bf q}^\prime},{\bf k}_4}c^\dagger_{{\bf q},\sigma}\left(1 - \hat n_{{{\bf q}^\prime},\sigma}\right) c_{{\bf k}_4,\sigma}\right)~.
\end{aligned}\end{equation}
where \({\bf q}\) and \({\bf q}^\prime\) are electronic states on the isoenergetic shells of energy \(\pm \varepsilon_j\). The change in occupancy of the state \({\bf q}\sigma\) from 0 to 1 leads to an excited state energy \(H_D = \varepsilon_{{\bf q}}\tau_{q\sigma} - \frac{1}{2}W_{{\bf q}}\left(\hat n_{{\bf q}, \uparrow} - \hat n_{{\bf q}, \uparrow}\right)^2 = \frac{1}{2}\varepsilon_{{\bf q}} - \frac{1}{2}W_{{\bf q}}\), where \(W_{{\bf q}}\) is a shorthand for \(W_{{\bf q},{\bf q},{\bf q},{\bf q}}\). We consider the first two terms on the right for the time being. The number operators \(\hat n_{{{\bf q}^\prime},\sigma}, \hat n_{{{\bf q}^\prime},\bar\sigma}\) project the initial state to that in which \({{\bf q}^\prime}\) is occupied, henceforth referred to as the particle sector (PS). The operator \(c^\dagger_{{\bf q},\sigma}\) can be combined with its Hermitian conjugate on the other side to produce a hole operator, leading to another projection onto the set of initial states in which the momentum state \({{\bf q}}\) is unoccupied, henceforth referred to as the hole sector (HS). Since the two terms are otherwise identical, their opposite signs lead to them cancelling each other. The remaining term involves the hole projection operators \(1 - \hat n_{{{\bf q}^\prime},\sigma}\) and \(1 - \hat n_{{{\bf q}},\sigma}\). Evaluating this term in the same way leads to
\begin{equation}\begin{aligned}\label{t2t61}
		T_2 G T_6^\dagger = -\sum_{\left\{{\bf k}_i\right\}, \sigma, \sigma^\prime} \sigma\sigma^\prime c^\dagger_{{\bf k}_1,\sigma^\prime}c_{{\bf k}_2,\sigma^\prime} c^\dagger_{{\bf k}_3,\sigma} c_{{\bf k}_4,\sigma}\sum_{{\bf q}, {\bf q}^\prime\in\text{HS}}\frac{W_{{\bf k}_1,{\bf k}_2,{\bf k}_3,{\bf q}} W_{{\bf q},{{\bf q}^\prime},{{\bf q}^\prime},{\bf k}_4}}{\omega - \frac{1}{2}\varepsilon_{{\bf q}} + \frac{1}{2}W_{{\bf q}}}  ~.
\end{aligned}\end{equation}
The particle-hole transformed term \(T_6^\dagger G T_2\) can be evaluated in the same way:
\begin{equation}\begin{aligned}\label{t6dagt2}
	T_6^\dagger G T_2 = \sum_{\left\{ {\bf k}_i \right\} }\sum_{\sigma,\sigma^\prime}\sum_{{\bf q}, {\bf q}^\prime} \left(W_{{{\bf q}^\prime},{{\bf q}^\prime},{\bf q},{\bf k}_4} \hat n_{{{\bf q}^\prime},\bar\sigma}c^\dagger_{{\bf q},\sigma}c_{{\bf k}_4,\sigma} - W_{{{\bf q}^\prime},{{\bf q}^\prime},{\bf q},{\bf k}_4}\hat n_{{{\bf q}^\prime},\sigma}c^\dagger_{{\bf q},\sigma}c_{{\bf k}_4,\sigma} - W_{{\bf q},{{\bf q}^\prime},{{\bf q}^\prime},{\bf k}_4}c^\dagger_{{\bf q},\sigma}\left(1 - \hat n_{{{\bf q}^\prime},\sigma}\right) c_{{\bf k}_4,\sigma}\right)\frac{1}{\omega - H_D}\times \\
	\sigma\sigma^\prime W_{{\bf k}_1,{\bf k}_2,{\bf k}_3,{\bf q}} c^\dagger_{{\bf k}_1, \sigma^\prime}c_{{\bf k}_2,\sigma^\prime}c^\dagger_{{\bf k}_3,\sigma}c_{{\bf q},\sigma}~.
\end{aligned}\end{equation}
For such a scattering process, the excited energy is given by \(H_D = \varepsilon_{{\bf q}}\tau_{q\sigma} - \frac{1}{2}W_{{\bf q}}\left(\hat n_{{\bf q}, \uparrow} - \hat n_{{\bf q}, \uparrow}\right)^2 = -\frac{1}{2}\varepsilon_{{\bf q}} - \frac{1}{2}W_{{\bf q}}\). The change of the sign in front of \(\varepsilon_{{\bf q}}\) arises from the fact that in this process, the state \({\bf q}\) is occupied in the intermediate excited state, owing to the \(c_{{\bf q},\sigma}\) operator to the right of the Greens function. Cancelling the first two terms in eq.~\ref{t6dagt2} (just as in the previous process) and evaluating the last term gives
\begin{equation}\begin{aligned}\label{t2t62}
	T_6^\dagger G T_2 &= \sum_{\left\{{\bf k}_i\right\}, \sigma\sigma^\prime}\sigma\sigma^\prime c^\dagger_{{\bf k}_1,\sigma^\prime}c_{{\bf k}_2,\sigma^\prime} c^\dagger_{{\bf k}_3,\sigma} c_{{\bf k}_4,\sigma}\sum_{{\bf q}\in\text{PS}\atop{{\bf q}^\prime \in \text{HS}}}\frac{W_{{\bf k}_1,{\bf k}_2,{\bf k}_3,{\bf q}} W_{{\bf q},{{\bf q}^\prime},{{\bf q}^\prime},{\bf k}_4}}{\omega + \frac{1}{2}\varepsilon_{{\bf q}} + \frac{1}{2}W_{{\bf q}}} \\
						  &= \sum_{\left\{{\bf k}_i\right\}, \sigma\sigma^\prime}\sigma\sigma^\prime c^\dagger_{{\bf k}_1,\sigma^\prime}c_{{\bf k}_2,\sigma^\prime} c^\dagger_{{\bf k}_3,\sigma} c_{{\bf k}_4,\sigma}\sum_{{\bf q},{\bf q}^\prime\in\text{HS}}\frac{W_{{\bf k}_1,{\bf k}_2,{\bf k}_3,{\bf q}_t} W_{{\bf q}_t,{{\bf q}^\prime},{{\bf q}^\prime},{\bf k}_4}}{\omega - \frac{1}{2}\varepsilon_{{\bf q}} + \frac{1}{2}W_{{\bf q}_t}}~.
\end{aligned}\end{equation}
At the last step, we exchanged the sum over the PS with a sum over the HS, leading to the transformation \({\bf q} \to {\bf q}_t\), where \({\bf q}_t = \left( \pi, \pi \right) - {\bf q}\) is the particle-hole transformed partner of the state at \({\bf q}\), and is obtained by reflecting the state \({\bf q}\) about the nearest nodal point in the Brillouin zone. These states are on isoenergetic contours that are at equal distance from but on opposite sides of the Fermi surface: \(\varepsilon_{{\bf q}} = -\varepsilon_{{\bf q}_t}\). 

Another pair of terms is obtained by taking the hermitian conjugate of the scattering terms, \(T_6 G T_2^\dagger\) and \(T_2^\dagger G T_6\). These are simply Hermitian conjugates of the two terms considered above:
\begin{equation}\begin{aligned}
		T_2^\dagger G T_6 = -\sum_{\left\{{\bf k}_i\right\}, \sigma\sigma^\prime}\sigma\sigma^\prime c^\dagger_{{\bf k}_1,\sigma^\prime}c_{{\bf k}_2,\sigma^\prime} c^\dagger_{{\bf k}_3,\sigma} c_{{\bf k}_4,\sigma}\sum_{{\bf q},{\bf q}^\prime\in\text{HS}}\frac{W_{{\bf q}_t,{\bf k}_4,{\bf k}_1,{\bf k}_2} W_{{\bf k}_3,{{\bf q}^\prime},{{\bf q}^\prime},{\bf q}_t}}{\omega - \frac{1}{2}\varepsilon_{{\bf q}} + \frac{1}{2}W_{{\bf q}_t}}\\
		T_6 G T_2^\dagger = \sum_{\left\{{\bf k}_i\right\}, \sigma, \sigma^\prime} \sigma\sigma^\prime c^\dagger_{{\bf k}_1,\sigma^\prime}c_{{\bf k}_2,\sigma^\prime} c^\dagger_{{\bf k}_3,\sigma} c_{{\bf k}_4,\sigma}\sum_{{\bf q}, {\bf q}^\prime\in\text{HS}}\frac{W_{{\bf q},{\bf k}_4,{\bf k}_1,{\bf k}_2} W_{{\bf k}_3,{{\bf q}^\prime},{{\bf q}^\prime},{\bf q}}}{\omega - \frac{1}{2}\varepsilon_{{\bf q}} + \frac{1}{2}W_{{\bf q}}}~.
\end{aligned}\end{equation}

\subsection{Scattering from the infrared subspace into the ultraviolet subspace}
We now consider the second term \(T_3^\dagger G T_3\):
\begin{equation}\begin{aligned}
	T_3^\dagger G T_3 &= \frac{1}{4}\sum_{\left\{ {\bf k}_i \right\} }\sum_{{\bf q},{\bf q}^\prime}\sum_{\sigma,\sigma^\prime} W_{{\bf q},{\bf k}_4, {\bf q}^\prime, {\bf k}_2} \sigma\sigma^\prime c^\dagger_{{\bf q},\sigma}c_{{\bf k}_4,\sigma}c^\dagger_{{\bf q}^\prime,\sigma^\prime}c_{{\bf k}_2,\sigma^\prime} \frac{1}{\omega - H_D} W_{{\bf k}_1,{\bf q}^\prime, {\bf k}_3, {\bf q}} \sigma\sigma^\prime c^\dagger_{{\bf k}_1,\sigma^\prime} c_{{\bf q}^\prime,\sigma^\prime} c^\dagger_{{\bf k}_3,\sigma} c_{{\bf q},\sigma}\\
			  &= \frac{1}{4}\sum_{\left\{ {\bf k}_i \right\}}\sum_{\sigma,\sigma^\prime}   c_{{\bf k}_4,\sigma} c_{{\bf k}_2,\sigma^\prime} c^\dagger_{{\bf k}_1,\sigma^\prime} c^\dagger_{{\bf k}_3,\sigma} \sum_{{\bf q},{\bf q}^\prime}\hat n_{{\bf q},\sigma} \hat n_{{\bf q}^\prime,\sigma^\prime}\frac{W_{{\bf q},{\bf k}_4, {\bf q}^\prime, {\bf k}_2} W_{{\bf k}_1,{\bf q}^\prime, {\bf k}_3, {\bf q}}}{\omega + \frac{1}{2}\varepsilon_{{\bf q}} + \frac{1}{2}W_{{\bf q}}}\\
			  &= \frac{1}{4}\sum_{\left\{ {\bf k}_i \right\}}\sum_{\sigma,\sigma^\prime} c^\dagger_{{\bf k}_1,\sigma^\prime} c_{{\bf k}_2,\sigma^\prime} c^\dagger_{{\bf k}_3,\sigma}c_{{\bf k}_4,\sigma}  \sum_{{\bf q}, {\bf q}^\prime \in \text{PS}}\frac{W_{{\bf q},{\bf k}_4, {\bf q}^\prime, {\bf k}_2} W_{{\bf k}_1,{\bf q}^\prime, {\bf k}_3, {\bf q}}}{\omega + \frac{1}{2}\varepsilon_{{\bf q}} + \frac{1}{2}W_{{\bf q}}}~,\\
			  &= \frac{1}{4}\sum_{\left\{ {\bf k}_i \right\}}\sum_{\sigma,\sigma^\prime} c^\dagger_{{\bf k}_1,\sigma^\prime} c_{{\bf k}_2,\sigma^\prime} c^\dagger_{{\bf k}_3,\sigma}c_{{\bf k}_4,\sigma}  \sum_{{\bf q}, {\bf q}^\prime \in \text{HS}}\frac{W_{{\bf q}_t,{\bf k}_4, {\bf q}_t^\prime, {\bf k}_2} W_{{\bf k}_1,{\bf q}_t^\prime, {\bf k}_3, {\bf q}_t}}{\omega - \frac{1}{2}\varepsilon_{{\bf q}} + \frac{1}{2}W_{{\bf q}_t}}~,
\end{aligned}\end{equation}
where \({\bf q}, {\bf q}^\prime\) are summed over all momentum states in the isoenergetic shell and in the particle sector (PS) (states that are occupied in the ground state). At the last step, we again replaced the sum over the particle sector with one over the hole sector by transforming \({\bf q},{{\bf q}^\prime} \to {\bf q}_t,{\bf q}_t^\prime\) and using \(\varepsilon_{{\bf q}_t} = -\varepsilon_{{\bf q}}\) in the denominator.

The particle-hole transformed term is \(T_3 G T_3^\dagger\):
\begin{equation}\begin{aligned}
	T_3 G T_3^\dagger &= \frac{1}{4}\sum_{\left\{ {\bf k}_i \right\} }\sum_{{\bf q},{\bf q}^\prime}\sum_{\sigma,\sigma^\prime} W_{{\bf k}_1,{\bf q}^\prime, {\bf k}_3, {\bf q}} \sigma\sigma^\prime c^\dagger_{{\bf k}_1,\sigma^\prime} c_{{\bf q}^\prime,\sigma^\prime} c^\dagger_{{\bf k}_3,\sigma} c_{{\bf q},\sigma} \frac{1}{\omega - H_D} W_{{\bf q},{\bf k}_4, {\bf q}^\prime, {\bf k}_2} \sigma\sigma^\prime c^\dagger_{{\bf q},\sigma}c_{{\bf k}_4,\sigma}c^\dagger_{{\bf q}^\prime,\sigma^\prime}c_{{\bf k}_2,\sigma^\prime}\\
			  &= \frac{1}{4}\sum_{\left\{ {\bf k}_i \right\}}\sum_{\sigma,\sigma^\prime} c^\dagger_{{\bf k}_1,\sigma^\prime} c_{{\bf k}_2,\sigma^\prime} c^\dagger_{{\bf k}_3,\sigma}c_{{\bf k}_4,\sigma} \sum_{{\bf q}, {\bf q}^\prime \in \text{HS}}\frac{W_{{\bf q},{\bf k}_4, {\bf q}^\prime, {\bf k}_2} W_{{\bf k}_1,{\bf q}^\prime, {\bf k}_3, {\bf q}}}{\omega - \frac{1}{2}\varepsilon_{{\bf q}} + \frac{1}{2}W_{{\bf q}}}~.
\end{aligned}\end{equation}

We can obtain two additional terms by switching the sequence of operators on the right hand side of the propagator:
\begin{equation}\begin{aligned}
	T_3^\dagger G \overline{T_3} &= \frac{1}{4}\sum_{\left\{ {\bf k}_i \right\} }\sum_{{\bf q},{\bf q}^\prime}\sum_{\sigma,\sigma^\prime} W_{{\bf q},{\bf k}_4, {\bf q}^\prime, {\bf k}_2} \sigma\sigma^\prime c^\dagger_{{\bf q},\sigma}c_{{\bf k}_4,\sigma}c^\dagger_{{\bf q}^\prime,\sigma^\prime}c_{{\bf k}_2,\sigma^\prime} \frac{1}{\omega - H_D} W_{{\bf k}_3, {\bf q}, {\bf k}_1,{\bf q}^\prime} \sigma\sigma^\prime c^\dagger_{{\bf k}_3,\sigma} c_{{\bf q},\sigma} c^\dagger_{{\bf k}_1,\sigma^\prime} c_{{\bf q}^\prime,\sigma^\prime}\\
			  &= \frac{1}{4}\sum_{\left\{ {\bf k}_i \right\}}\sum_{\sigma,\sigma^\prime} c^\dagger_{{\bf k}_1,\sigma^\prime} c_{{\bf k}_2,\sigma^\prime} c^\dagger_{{\bf k}_3,\sigma}c_{{\bf k}_4,\sigma}  \sum_{{\bf q}, {\bf q}^\prime \in \text{HS}}\frac{W_{{\bf q}_t,{\bf k}_4, {\bf q}_t^\prime, {\bf k}_2} W_{{\bf k}_3, {\bf q}_t, {\bf k}_1,{\bf q}_t^\prime}}{\omega - \frac{1}{2}\varepsilon_{{\bf q}} + \frac{1}{2}W_{{\bf q}_t}}~,
\end{aligned}\end{equation}
\begin{equation}\begin{aligned}
	T_3 G \overline T_3^\dagger &= \frac{1}{4}\sum_{\left\{ {\bf k}_i \right\} }\sum_{{\bf q},{\bf q}^\prime}\sum_{\sigma,\sigma^\prime} W_{{\bf k}_1,{\bf q}^\prime, {\bf k}_3, {\bf q}} \sigma\sigma^\prime c^\dagger_{{\bf k}_1,\sigma^\prime} c_{{\bf q}^\prime,\sigma^\prime} c^\dagger_{{\bf k}_3,\sigma} c_{{\bf q},\sigma} \frac{1}{\omega - H_D} W_{{\bf q}^\prime, {\bf k}_2, {\bf q},{\bf k}_4} \sigma\sigma^\prime c^\dagger_{{\bf q}^\prime,\sigma^\prime}c_{{\bf k}_2,\sigma^\prime} c^\dagger_{{\bf q},\sigma}c_{{\bf k}_4,\sigma}\\
			  &= \frac{1}{4}\sum_{\left\{ {\bf k}_i \right\}}\sum_{\sigma,\sigma^\prime} c^\dagger_{{\bf k}_1,\sigma^\prime} c_{{\bf k}_2,\sigma^\prime} c^\dagger_{{\bf k}_3,\sigma}c_{{\bf k}_4,\sigma}  \sum_{{\bf q}, {\bf q}^\prime \in \text{HS}}\frac{W_{{\bf q}^\prime, {\bf k}_2, {\bf q},{\bf k}_4} W_{{\bf k}_1,{\bf q}^\prime, {\bf k}_3, {\bf q}}}{\omega - \frac{1}{2}\varepsilon_{{\bf q}} + \frac{1}{2}W_{{\bf q}}}~.
\end{aligned}\end{equation}

\subsection{Coherent simultaneous scattering within the infrared and ultraviolet subspaces}
The remaining sets of terms that we need to consider are:
\begin{equation}\begin{aligned}
	T_4 G T_4^\dagger &= \frac{1}{4}\sum_{\left\{ {\bf k}_i \right\} }\sum_{{\bf q},{\bf q}^\prime}\sum_{\sigma,\sigma^\prime}  W_{{\bf q},{\bf k}_4, {\bf k}_1, {\bf q}^\prime} \sigma\sigma^\prime c^\dagger_{{\bf q},\sigma}c_{{\bf k}_4,\sigma}c^\dagger_{{\bf k}_1,\sigma^\prime}c_{{\bf q}^\prime,\sigma^\prime} \frac{1}{\omega - H_D}W_{{\bf q}^\prime, {\bf k}_2, {\bf k}_3, {\bf q}} \sigma\sigma^\prime c^\dagger_{{\bf q}^\prime,\sigma^\prime}c_{{\bf k}_2,\sigma^\prime}c^\dagger_{{\bf k}_3,\sigma}c_{{\bf q},\sigma}\\
		  &= \frac{1}{4}\sum_{\left\{ {\bf k}_i \right\} }\sum_{{\bf q},{\bf q}^\prime}\sum_{\sigma,\sigma^\prime} \hat n_{{\bf q},\sigma} \left(1 - \hat n_{{\bf q}^\prime,\sigma^\prime}\right) c_{{\bf k}_4,\sigma}c^\dagger_{{\bf k}_1,\sigma^\prime}c_{{\bf k}_2,\sigma^\prime}c^\dagger_{{\bf k}_3,\sigma} \frac{W_{{\bf q},{\bf k}_4, {\bf k}_1, {\bf q}^\prime} W_{{\bf q}^\prime, {\bf k}_2, {\bf k}_3, {\bf q}}}{\omega + \frac{1}{2}\varepsilon_{{\bf q}}  + \frac{1}{2}W_{{\bf q}}}  \\
		  &= -\frac{1}{4}\sum_{\left\{ {\bf k}_i \right\} }\sum_{\sigma,\sigma^\prime} c^\dagger_{{\bf k}_1,\sigma^\prime}c_{{\bf k}_2,\sigma^\prime}c^\dagger_{{\bf k}_3,\sigma}c_{{\bf k}_4,\sigma} \sum_{{\bf q}, {\bf q}^\prime \in \text{HS}}\frac{W_{{\bf q}_t,{\bf k}_4, {\bf k}_1, {\bf q}^\prime} W_{{\bf q}^\prime, {\bf k}_2, {\bf k}_3, {\bf q}_t}}{\omega - \frac{1}{2}\varepsilon_{{\bf q}}  + \frac{1}{2}W_{{\bf q}_t}}~.
\end{aligned}\end{equation}

\begin{equation}\begin{aligned}
	T_5 G T_5^\dagger &= \frac{1}{4}\sum_{\left\{ {\bf k}_i \right\} }\sum_{{\bf q},{\bf q}^\prime}\sum_{\sigma,\sigma^\prime} W_{{\bf k}_1 {\bf q}^\prime, {\bf q}, {\bf k}_4} \sigma\sigma^\prime c^\dagger_{{\bf k}_1,\sigma^\prime}c_{{\bf q}^\prime,\sigma^\prime}c^\dagger_{{\bf q},\sigma}c_{{\bf k}_4,\sigma} \frac{1}{\omega - H_D}W_{{\bf k}_3, {\bf q}, {\bf q}^\prime, {\bf k}_2} \sigma\sigma^\prime c^\dagger_{{\bf k}_3,\sigma}c_{{\bf q},\sigma}c^\dagger_{{\bf q}^\prime,\sigma^\prime}c_{{\bf k}_2,\sigma^\prime}\\
		  &= \frac{1}{4}\sum_{\left\{ {\bf k}_i \right\} }\sum_{{\bf q},{\bf q}^\prime}\sum_{\sigma,\sigma^\prime} \hat n_{{\bf q},\sigma} \left(1 - \hat n_{{\bf q}^\prime,\sigma^\prime}\right) c^\dagger_{{\bf k}_1,\sigma^\prime}c_{{\bf k}_4,\sigma}c^\dagger_{{\bf k}_3,\sigma}c_{{\bf k}_2,\sigma^\prime} \frac{W_{{\bf k}_1, {\bf q}^\prime, {\bf q},{\bf k}_4} W_{{\bf k}_3, {\bf q}, {\bf q}^\prime, {\bf k}_2}}{\omega + \frac{1}{2}\varepsilon_{{\bf q}}  + \frac{1}{2}W_{{\bf q}}}  \\
		  &= -\frac{1}{4}\sum_{\left\{ {\bf k}_i \right\} }\sum_{\sigma,\sigma^\prime} c^\dagger_{{\bf k}_1,\sigma^\prime} c_{{\bf k}_2,\sigma^\prime} c^\dagger_{{\bf k}_3,\sigma} c_{{\bf k}_4,\sigma} \sum_{{\bf q}, {\bf q}^\prime \in \text{HS}}\frac{W_{{\bf k}_1, {\bf q}^\prime, {\bf q}_t,{\bf k}_4} W_{{\bf k}_3, {\bf q}_t, {\bf q}^\prime, {\bf k}_2}}{\omega - \frac{1}{2}\varepsilon_{{\bf q}}  + \frac{1}{2}W_{{\bf q}_t}}~.
\end{aligned}\end{equation}

\begin{equation}\begin{aligned}
	T_5^\dagger G T_4 &= \frac{1}{4}\sum_{\left\{ {\bf k}_i \right\} }\sum_{{\bf q},{\bf q}^\prime}\sum_{\sigma,\sigma^\prime} W_{{\bf k}_3, {\bf q}, {\bf q}^\prime, {\bf k}_2} \sigma\sigma^\prime c^\dagger_{{\bf k}_3,\sigma}c_{{\bf q},\sigma}c^\dagger_{{\bf q}^\prime,\sigma^\prime}c_{{\bf k}_2,\sigma^\prime} \frac{1}{\omega - H_D}W_{{\bf q},{\bf k}_4, {\bf k}_1, {\bf q}^\prime} \sigma\sigma^\prime c^\dagger_{{\bf q},\sigma}c_{{\bf k}_4,\sigma}c^\dagger_{{\bf k}_1,\sigma^\prime}c_{{\bf q}^\prime,\sigma^\prime}\\
		  &= -\frac{1}{4}\sum_{\left\{ {\bf k}_i \right\} }\sum_{{\bf q},{\bf q}^\prime}\sum_{\sigma,\sigma^\prime} \hat n_{{\bf q}^\prime,\sigma^\prime} \left(1 - \hat n_{{\bf q},\sigma}\right) c^\dagger_{{\bf k}_3,\sigma}c_{{\bf k}_2,\sigma^\prime}c_{{\bf k}_4,\sigma} c^\dagger_{{\bf k}_1,\sigma^\prime} \frac{W_{{\bf q},{\bf k}_4, {\bf k}_1, {\bf q}^\prime} W_{{\bf k}_3, {\bf q}, {\bf q}^\prime, {\bf k}_2}}{\omega - \frac{1}{2}\varepsilon_{{\bf q}}  + \frac{1}{2}W_{{\bf q}}}  \\
		  &= -\frac{1}{4}\sum_{\left\{ {\bf k}_i \right\}}\sum_{\sigma,\sigma^\prime} c^\dagger_{{\bf k}_1,\sigma^\prime}c_{{\bf k}_2,\sigma^\prime}c^\dagger_{{\bf k}_3,\sigma} c_{{\bf k}_4,\sigma} \sum_{{\bf q}, {\bf q}^\prime \in \text{HS}}\frac{W_{{\bf q},{\bf k}_4, {\bf k}_1, {\bf q}_t^\prime} W_{{\bf k}_3, {\bf q}, {\bf q}_t^\prime, {\bf k}_2}}{\omega - \frac{1}{2}\varepsilon_{{\bf q}}  + \frac{1}{2}W_{{\bf q}}}~.
\end{aligned}\end{equation}

\begin{equation}\begin{aligned}
	T_4^\dagger G T_5 &= \frac{1}{4}\sum W_{{\bf q}^\prime, {\bf k}_2, {\bf k}_3, {\bf q}} \sigma\sigma^\prime c^\dagger_{{\bf q}^\prime,\sigma^\prime}c_{{\bf k}_2,\sigma^\prime}c^\dagger_{{\bf k}_3,\sigma}c_{{\bf q},\sigma} \frac{1}{\omega - H_D} W_{{\bf k}_1 {\bf q}^\prime, {\bf q}, {\bf k}_4} \sigma\sigma^\prime c^\dagger_{{\bf k}_1,\sigma^\prime}c_{{\bf q}^\prime,\sigma^\prime}c^\dagger_{{\bf q},\sigma}c_{{\bf k}_4,\sigma}\\
		  &= -\frac{1}{4}\sum_{\left\{ {\bf k}_i \right\} }\sum_{{\bf q},{\bf q}^\prime}\sum_{\sigma,\sigma^\prime} \hat n_{{\bf q}^\prime,\sigma^\prime} \left(1 - \hat n_{{\bf q},\sigma}\right) c_{{\bf k}_2,\sigma^\prime} c^\dagger_{{\bf k}_3,\sigma} c^\dagger_{{\bf k}_1,\sigma^\prime}c_{{\bf k}_4,\sigma}  \frac{W_{{\bf q}^\prime, {\bf k}_2, {\bf k}_3, {\bf q}} W_{{\bf k}_1 {\bf q}^\prime, {\bf q}, {\bf k}_4}}{\omega - \frac{1}{2}\varepsilon_{{\bf q}}  + \frac{1}{2}W_{{\bf q}}}  \\
		  &= -\frac{1}{4}\sum_{\left\{ {\bf k}_i \right\}}\sum_{\sigma,\sigma^\prime} c^\dagger_{{\bf k}_1,\sigma^\prime}c_{{\bf k}_2,\sigma^\prime}c^\dagger_{{\bf k}_3,\sigma} c_{{\bf k}_4,\sigma} \sum_{{\bf q}, {\bf q}^\prime \in \text{HS}}\frac{W_{{\bf q}_t^\prime, {\bf k}_2, {\bf k}_3, {\bf q}} W_{{\bf k}_1 {\bf q}_t^\prime, {\bf q}, {\bf k}_4}}{\omega - \frac{1}{2}\varepsilon_{{\bf q}}  + \frac{1}{2}W_{{\bf q}}}~.
\end{aligned}\end{equation}

\subsection{Net renormalisation for the bath correlation term}
Adding contributions from all these terms, the total renormalisation in the Hamiltonian in the context of \(W\) comes out to be
\begin{equation}\begin{aligned}
	\Delta H\bigg|_{W} = \frac{1}{4}\sum_{\left\{ {\bf k}_i \right\}}\sum_{\sigma,\sigma^\prime} c^\dagger_{{\bf k}_1,\sigma^\prime}c_{{\bf k}_2,\sigma^\prime}c^\dagger_{{\bf k}_3,\sigma} c_{{\bf k}_4,\sigma} \sum_{{\bf q}, {\bf q}^\prime \in \text{HS}}\left[-\frac{W_{{\bf k}_1,{\bf k}_2,{\bf k}_3,{\bf q}} W_{{\bf q},{{\bf q}^\prime},{{\bf q}^\prime},{\bf k}_4}}{\omega - \frac{1}{2}\varepsilon_{{\bf q}} + \frac{1}{2}W_{{\bf q}}} + \frac{W_{{\bf k}_1,{\bf k}_2,{\bf k}_3,{\bf q}_t} W_{{\bf q}_t,{{\bf q}^\prime},{{\bf q}^\prime},{\bf k}_4}}{\omega - \frac{1}{2}\varepsilon_{{\bf q}} + \frac{1}{2}W_{{\bf q}_t}} \right.\\
\left.- \frac{W_{{\bf q}_t,{\bf k}_4,{\bf k}_1,{\bf k}_2} W_{{\bf k}_3,{{\bf q}^\prime},{{\bf q}^\prime},{\bf q}_t}}{\omega - \frac{1}{2}\varepsilon_{{\bf q}} + \frac{1}{2}W_{{\bf q}_t}}
 + \frac{W_{{\bf q},{\bf k}_4,{\bf k}_1,{\bf k}_2} W_{{\bf k}_3,{{\bf q}^\prime},{{\bf q}^\prime},{\bf q}}}{\omega - \frac{1}{2}\varepsilon_{{\bf q}} + \frac{1}{2}W_{{\bf q}}}
 + \frac{W_{{\bf q}_t,{\bf k}_4, {\bf q}_t^\prime, {\bf k}_2} W_{{\bf k}_1,{\bf q}_t^\prime, {\bf k}_3, {\bf q}_t}}{\omega - \frac{1}{2}\varepsilon_{{\bf q}} + \frac{1}{2}W_{{\bf q}_t}} + \frac{W_{{\bf q},{\bf k}_4, {\bf q}^\prime, {\bf k}_2} W_{{\bf k}_1,{\bf q}^\prime, {\bf k}_3, {\bf q}}}{\omega - \frac{1}{2}\varepsilon_{{\bf q}} + \frac{1}{2}W_{{\bf q}}}\right.\\
\left.+ \frac{W_{{\bf q}_t,{\bf k}_4, {\bf q}_t^\prime, {\bf k}_2} W_{{\bf k}_3, {\bf q}_t, {\bf k}_1,{\bf q}_t^\prime}}{\omega - \frac{1}{2}\varepsilon_{{\bf q}} + \frac{1}{2}W_{{\bf q}_t}} 
+ \frac{W_{{\bf q}^\prime, {\bf k}_2, {\bf q},{\bf k}_4} W_{{\bf k}_1,{\bf q}^\prime, {\bf k}_3, {\bf q}}}{\omega - \frac{1}{2}\varepsilon_{{\bf q}} + \frac{1}{2}W_{{\bf q}}} - \frac{W_{{\bf q}_t,{\bf k}_4, {\bf k}_1, {\bf q}^\prime} W_{{\bf q}^\prime, {\bf k}_2, {\bf k}_3, {\bf q}_t}}{\omega - \frac{1}{2}\varepsilon_{{\bf q}}  + \frac{1}{2}W_{{\bf q}_t}}
- \frac{W_{{\bf k}_1, {\bf q}^\prime, {\bf q}_t,{\bf k}_4} W_{{\bf k}_3, {\bf q}_t, {\bf q}^\prime, {\bf k}_2}}{\omega - \frac{1}{2}\varepsilon_{{\bf q}}  + \frac{1}{2}W_{{\bf q}_t}} \right.\\
\left.- \frac{W_{{\bf q},{\bf k}_4, {\bf k}_1, {\bf q}_t^\prime} W_{{\bf k}_3, {\bf q}, {\bf q}_t^\prime, {\bf k}_2}}{\omega - \frac{1}{2}\varepsilon_{{\bf q}}  + \frac{1}{2}W_{{\bf q}}} - \frac{W_{{\bf q}_t^\prime, {\bf k}_2, {\bf k}_3, {\bf q}} W_{{\bf k}_1 {\bf q}_t^\prime, {\bf q}, {\bf k}_4}}{\omega - \frac{1}{2}\varepsilon_{{\bf q}}  + \frac{1}{2}W_{{\bf q}}} \right]~.
\end{aligned}\end{equation}
For both the forms of \(W_{\left\{ {\bf k}_i \right\} }\) that we are considering here (and mentioned below eq.~\ref{hamiltonian}), the couplings \(W_{{\bf q}}\) and \(W_{{\bf q}_t}\) are equal, where \({\bf q}_t = \vec \pi - {\bf q}\). Moreover, we have the following relations between the scattering vertex strengths:
\begin{equation}\begin{aligned}
	W_{{\bf q}_t,{\bf k}_4, {\bf k}_1, {\bf q}^\prime} = W_{{\bf k}_1, {\bf q}^\prime, {\bf q}_t,{\bf k}_4}, \quad W_{{\bf q}^\prime, {\bf k}_2, {\bf k}_3, {\bf q}_t} = W_{{\bf k}_3, {\bf q}_t, {\bf q}^\prime, {\bf k}_2}~,\quad W_{{\bf q},{\bf k}_4, {\bf q}^\prime, {\bf k}_2}=W_{{\bf q}^\prime, {\bf k}_2,{\bf q},{\bf k}_4},\quad W_{{\bf k}_1,{\bf q}_t^\prime, {\bf k}_3, {\bf q}_t}=W_{{\bf k}_3, {\bf q}_t,{\bf k}_1,{\bf q}_t^\prime}
\end{aligned}\end{equation}
leading to the simplified expression
\begin{equation}\begin{aligned}
	\Delta H\bigg|_{W} = \frac{1}{4}\sum_{\left\{ {\bf k}_i \right\}}\sum_{\sigma,\sigma^\prime} c^\dagger_{{\bf k}_1,\sigma^\prime}c_{{\bf k}_2,\sigma^\prime}c^\dagger_{{\bf k}_3,\sigma} c_{{\bf k}_4,\sigma} \sum_{{\bf q}, {\bf q}^\prime \in \text{HS}}\left[-\frac{W_{{\bf k}_1,{\bf k}_2,{\bf k}_3,{\bf q}} W_{{\bf q},{{\bf q}^\prime},{{\bf q}^\prime},{\bf k}_4}}{\omega - \frac{1}{2}\varepsilon_{{\bf q}} + \frac{1}{2}W_{{\bf q}}} + \frac{W_{{\bf k}_1,{\bf k}_2,{\bf k}_3,{\bf q}_t} W_{{\bf q}_t,{{\bf q}^\prime},{{\bf q}^\prime},{\bf k}_4}}{\omega - \frac{1}{2}\varepsilon_{{\bf q}} + \frac{1}{2}W_{{\bf q}}} \right.\\
\left.- \frac{W_{{\bf q}_t,{\bf k}_4,{\bf k}_1,{\bf k}_2} W_{{\bf k}_3,{{\bf q}^\prime},{{\bf q}^\prime},{\bf q}_t}}{\omega - \frac{1}{2}\varepsilon_{{\bf q}} + \frac{1}{2}W_{{\bf q}}}
 + \frac{W_{{\bf q},{\bf k}_4,{\bf k}_1,{\bf k}_2} W_{{\bf k}_3,{{\bf q}^\prime},{{\bf q}^\prime},{\bf q}}}{\omega - \frac{1}{2}\varepsilon_{{\bf q}} + \frac{1}{2}W_{{\bf q}}}
 + \frac{2W_{{\bf q}_t,{\bf k}_4, {\bf q}_t^\prime, {\bf k}_2} W_{{\bf k}_1,{\bf q}_t^\prime, {\bf k}_3, {\bf q}_t}}{\omega - \frac{1}{2}\varepsilon_{{\bf q}} + \frac{1}{2}W_{{\bf q}}} + \frac{2W_{{\bf q},{\bf k}_4, {\bf q}^\prime, {\bf k}_2} W_{{\bf k}_1,{\bf q}^\prime, {\bf k}_3, {\bf q}}}{\omega - \frac{1}{2}\varepsilon_{{\bf q}} + \frac{1}{2}W_{{\bf q}}}\right.\\
\left. - \frac{2W_{{\bf q}_t,{\bf k}_4, {\bf k}_1, {\bf q}^\prime} W_{{\bf q}^\prime, {\bf k}_2, {\bf k}_3, {\bf q}_t}}{\omega - \frac{1}{2}\varepsilon_{{\bf q}}  + \frac{1}{2}W_{{\bf q}}} - \frac{2W_{{\bf q},{\bf k}_4, {\bf k}_1, {\bf q}_t^\prime} W_{{\bf k}_3, {\bf q}, {\bf q}_t^\prime, {\bf k}_2}}{\omega - \frac{1}{2}\varepsilon_{{\bf q}}  + \frac{1}{2}W_{{\bf q}}}\right]~.
\end{aligned}\end{equation}




\section{Renormalisation of the Kondo scattering term \(J\)}
We take a closer look at the Kondo scattering terms \(T_1\) and \(T_7\) in \(H_X\):
\begin{equation}\begin{aligned}
	T_1^\dagger + T_7 =& \frac{1}{2}\underbrace{\sum_{{\bf k},{\bf q},\sigma}J_{{\bf k},{\bf q}}S_d^z \sigma c^\dagger_{{\bf q},\sigma} c_{{\bf k},\sigma}}_{T_{1,z}^\dagger} + \frac{1}{2}\underbrace{\sum_{{\bf k},{\bf q}}J_{{\bf k},{\bf q}} S_d^+ c^\dagger_{{\bf q},\downarrow} c_{{\bf k},\uparrow}}_{T_{1,+-}^\dagger} + \frac{1}{2}\underbrace{\sum_{{\bf k},{\bf q}}S_d^- c^\dagger_{{\bf q},\uparrow} c_{{\bf k},\downarrow}}_{T_{1,-+}^\dagger} \\
			   &\frac{1}{2}\underbrace{\sum_{{\bf q}^\prime,{\bf q},\sigma}J_{{\bf q}^\prime,{\bf q}}S_d^z \sigma c^\dagger_{{\bf q},\sigma} c_{{\bf q}^\prime,\sigma}}_{T_{7,z}^\dagger} + \frac{1}{2}\underbrace{\sum_{{\bf q}^\prime,{\bf q}}J_{{\bf q}^\prime,{\bf q}} S_d^+ c^\dagger_{{\bf q},\downarrow} c_{{\bf q}^\prime,\uparrow}}_{T_{7,+-}^\dagger} + \frac{1}{2}\underbrace{\sum_{{\bf q}^\prime,{\bf q}}S_d^- c^\dagger_{{\bf q},\uparrow} c_{{\bf q}^\prime,\downarrow}}_{T_{7,-+}^\dagger}
\end{aligned}\end{equation}
We note that scattering processes involving the pairs \(\left(T_{1,\pm \mp}^\dagger, T_{1,\pm \mp}\right)\) and \(\left(T_{7,z}, T_6\right) \) will renormalise the \(S_d^z\) term, while those involving the pairs \(\left(T_{1,z}, T_{1,\pm \mp}\right)\) and \(\left(T_{7,\pm \mp}, T_6\right)\) will renormalise the \(S_d^\pm\) terms. 

\subsection{Impurity-mediated spin-flip scattering purely through Kondo-like processes}
We first consider the renormalisation to the \(S_d^z\) term, arising purely from the Kondo terms:
\begin{equation}\begin{aligned}
	\sum_{\sigma=\pm}T_{1,\sigma\bar\sigma}^\dagger G T_{1,\sigma\bar\sigma} = \sum_{\sigma=\pm}\frac{1}{4}\sum_{{\bf k}_2,{\bf q}}J_{{\bf k}_2,{\bf q}} S_d^\sigma c^\dagger_{{\bf q},\bar\sigma} c_{{\bf k}_2,\sigma} \frac{1}{\omega - H_D}\sum_{{\bf k}_1,{\bf q}}J_{{\bf k}_1,{\bf q}} S_d^{\bar\sigma} c^\dagger_{{\bf k}_1,\sigma} c_{{\bf q},\bar\sigma}~,
\end{aligned}\end{equation}
where \(c_{k,+(-)} \equiv c_{k,\uparrow(\downarrow)}\). The excitation energy for such processes is given by \(H_D = |\varepsilon_j| - J_{{\bf q}}/4 - W_{{\bf q}}/2\), due to the fact that the impurity spin flip and the spin flip of the conduction bath state \({\bf q}\) occurs in an anti-parallel fashion. Substituting this, performing the usual contraction and projection of the state \({\bf q}\) and carrying out the spin manipulation \(S_d^\sigma S_d^{\bar\sigma} = \frac{1}{2} + \sigma S_d^z\) results in the expression
\begin{equation}\begin{aligned}\label{t1t1p}
	\sum_{\sigma=\pm}T_{1,\sigma\bar\sigma}^\dagger G T_{1,\sigma\bar\sigma} &= \sum_{\sigma=\pm}\frac{1}{4}\sum_{{\bf k}_2,{\bf k}_1} \left(\frac{1}{2} + \sigma S_d^z\right) c_{{\bf k}_2,\sigma} c^\dagger_{{\bf k}_1,\sigma}\sum_{{\bf q}}  \hat n_{{\bf q},\bar\sigma} \frac{J_{{\bf k}_2,{\bf q}} J_{{\bf k}_1,{\bf q}}}{\omega - |\varepsilon_j| + J_{{\bf q}}/4 + W_{{\bf q}}/2}\\
										 &= -\frac{1}{2}\sum_{{\bf k}_2,{\bf k}_1} S_d^z \sigma \frac{1}{2}c^\dagger_{{\bf k}_1,\sigma}c_{{\bf k}_2,\sigma} \sum_{{\bf q} \in \text{PS}} \frac{J_{{\bf k}_2,{\bf q}} J_{{\bf k}_1,{\bf q}}}{\omega - |\varepsilon_j| + J_{{\bf q}}/4 + W_{{\bf q}}/2} + \bigg[\text{pot. scatt. terms}\bigg]~.
\end{aligned}\end{equation}
The particle-hole exchanged partner is
\begin{equation}\begin{aligned}\label{t1t1h}
	\sum_{\sigma=\pm}T_{1,\sigma\bar\sigma} G T^\dagger_{1,\sigma\bar\sigma} &= \sum_{\sigma=\pm}\frac{1}{4} \sum_{{\bf k}_1,{\bf q}}J_{{\bf k}_1,{\bf q}} S_d^{\bar\sigma} c^\dagger_{{\bf k}_1,\sigma} c_{{\bf q},\bar\sigma} \frac{1}{\omega - H_D}\sum_{{\bf k}_2,{\bf q}}J_{{\bf k}_2,{\bf q}} S_d^\sigma c^\dagger_{{\bf q},\bar\sigma} c_{{\bf k}_2,\sigma}\\
										 &= \sum_{\sigma=\pm}\frac{1}{4}\sum_{{\bf k}_2,{\bf k}_1} \left(\frac{1}{2} + \bar\sigma S_d^z\right) c^\dagger_{{\bf k}_1,\sigma} c_{{\bf k}_2,\sigma} \sum_{{\bf q}} \left(1 - \hat n_{{\bf q},\bar\sigma}\right) \frac{J_{{\bf k}_2,{\bf q}} J_{{\bf k}_1,{\bf q}}}{\omega - |\varepsilon_j| + J_{{\bf q}}/4 + W_{{\bf q}}/2}\\
										 &= -\frac{1}{2}\sum_{{\bf k}_2,{\bf k}_1, \sigma} S_d^z \sigma \frac{1}{2}c^\dagger_{{\bf k}_1,\sigma}c_{{\bf k},\sigma} \sum_{{\bf q} \in \text{HS}} \frac{J_{{\bf k}_2,{\bf q}} J_{{\bf k}_1,{\bf q}}}{\omega - |\varepsilon_j| + J_{{\bf q}}/4 + W_{{\bf q}}/2} + \bigg[\text{pot. scatt. terms}\bigg]\\
										 &= -\frac{1}{2}\sum_{{\bf k}_2,{\bf k}_1, \sigma} S_d^z \sigma \frac{1}{2}c^\dagger_{{\bf k}_1,\sigma}c_{{\bf k},\sigma} \sum_{{\bf q} \in \text{PS}} \frac{J_{{\bf k}_2,{\bf q}} J_{{\bf k}_1,{\bf q}}}{\omega - |\varepsilon_j| + J_{{\bf q}}/4 + W_{{\bf q}}/2} + \bigg[\text{pot. scatt. terms}\bigg]~.
\end{aligned}\end{equation}
At the last step, we converted the sum over the hole sector into one over the particle sector. As argued elsewhere, transforming from the particle to the hole sector results in the inversion of the sign of the factor \(\left[\cos\left( aq_1^x \right) - \cos\left( aq_1^y \right) \right]\), leaving the product \(J_{{\bf k}_2,{\bf q}} J_{{\bf k}_1,{\bf q}}\) unchanged. The couplings in the denominators involve the product of four such factors, so they also remain unchanged.

\subsection{Scattering processes involving the Kondo interaction as well as the bath interaction}
Next, we consider scattering processes involving \(T_7, T_6\):
\begin{equation}\begin{aligned}\label{t7t6}
	T_{7,z} G T_6 &= \frac{1}{4}\sum_{{\bf k}_1,{\bf k}_2,{\bf q},\sigma}\sigma J_{-{\bf q},{\bf q}}S_d^z c^\dagger_{{\bf q},\sigma} c_{-{\bf q},\sigma} \frac{1}{\omega - H_D} \left(- W_{{\bf q},-{\bf q},{\bf k}_1, {\bf k}_2}\right) \left(2 c^\dagger_{k_1\sigma}c_{k_2\sigma}c^\dagger_{-{\bf q},\sigma}c_{{\bf q},\sigma} - 2 c^\dagger_{k_1\bar\sigma}c_{k_2\bar\sigma}c^\dagger_{-{\bf q},\sigma}c_{{\bf q},\sigma}\right)\\
			      &= -\frac{1}{2}\sum_{{\bf k}_1,{\bf k}_2,{\bf q},\sigma}\sigma S_d^z \left(c^\dagger_{k_1\sigma}c_{k_2\sigma} - c^\dagger_{k_1\bar\sigma}c_{k_2\bar\sigma}\right) \sum_{{\bf q} \in \text{PS}}\frac{J_{-{\bf q},{\bf q}} W_{{\bf q},-{\bf q},{\bf k}_1, {\bf k}_2}}{\omega - |\varepsilon_j| + J_{{\bf q}}/4 + W_{{\bf q}}/2} \\
			      &= -2\sum_{{\bf k}_1,{\bf k}_2,\sigma}\sigma S_d^z \frac{1}{2}c^\dagger_{k_1\sigma}c_{k_2\sigma} \sum_{{\bf q} \in \text{PS}}\frac{J_{-{\bf q},{\bf q}} W_{{\bf q},-{\bf q},{\bf k}_1, {\bf k}_2}}{\omega - |\varepsilon_j| + J_{{\bf q}}/4 + W_{{\bf q}}/2} 
\end{aligned}\end{equation}
Another scattering process is obtained by switching the operators \(T_7\) and \(T_6\):
\begin{equation}\begin{aligned}\label{t6t7}
	T_{6}^\dagger G T_{7,z} &= \frac{1}{4}\sum_{{\bf k}_1,{\bf k}_2,{\bf q},\sigma} \left(- W_{{\bf q},-{\bf q},{\bf k}_1, {\bf k}_2}\right) \left(2 c^\dagger_{k_1\sigma}c_{k_2\sigma}c^\dagger_{-{\bf q},\sigma}c_{{\bf q},\sigma} - 2 c^\dagger_{k_1\bar\sigma}c_{k_2\bar\sigma}c^\dagger_{-{\bf q},\sigma}c_{{\bf q},\sigma}\right) \frac{1}{\omega - H_D} \sigma J_{-{\bf q},{\bf q}}S_d^z c^\dagger_{{\bf q},\sigma} c_{-{\bf q},\sigma}\\
			      &= -\frac{1}{2}\sum_{{\bf k}_1,{\bf k}_2,{\bf q},\sigma}\sigma S_d^z \left(c^\dagger_{k_1\sigma}c_{k_2\sigma} - c^\dagger_{k_1\bar\sigma}c_{k_2\bar\sigma}\right) \sum_{{\bf q} \in \text{HS}}\frac{J_{-{\bf q},{\bf q}} W_{{\bf q},-{\bf q},{\bf k}_1, {\bf k}_2}}{\omega - |\varepsilon_j| + J_{{\bf q}}/4 + W_{{\bf q}}/2} \\
			      &= -2\sum_{{\bf k}_1,{\bf k}_2,\sigma}\sigma S_d^z \frac{1}{2} c^\dagger_{k_1\sigma}c_{k_2\sigma} \sum_{{\bf q} \in \text{HS}}\frac{J_{-{\bf q},{\bf q}} W_{{\bf q},-{\bf q},{\bf k}_1, {\bf k}_2}}{\omega - |\varepsilon_j| + J_{{\bf q}}/4 + W_{{\bf q}}/2} \\
			      &= -2\sum_{{\bf k}_1,{\bf k}_2,\sigma}\sigma S_d^z \frac{1}{2} c^\dagger_{k_1\sigma}c_{k_2\sigma} \sum_{{\bf q} \in \text{PS}}\frac{J_{-{\bf q},{\bf q}} W_{{\bf q},-{\bf q},{\bf k}_1, {\bf k}_2}}{\omega - |\varepsilon_j| + J_{{\bf q}}/4 + W_{{\bf q}}/2} \\
\end{aligned}\end{equation}
At the last step, we again transformed from the particle sector to the hole sector using the arguments mentioned just above.

\subsection{Net renormalisation to the Kondo interaction}
Owing to spin-rotation symmetry of the Kondo interaction, it suffices to calculate the renormalisation for the Ising-like interaction term, since that of the spin-flip terms will be equal to it. Upon adding the Hamltonian renormalisation from the four classes eqs.~\ref{t1t1p}, \ref{t1t1h}, \ref{t7t6} and \ref{t6t7}, the total renormalisation in the Ising part of the Kondo interaction comes out to be
\begin{equation}\begin{aligned}
	\Delta J^{(j)}_{{\bf k}_1, {\bf k}_2} = -\sum_{{\bf q} \in \text{PS}} \frac{J^{(j)}_{{\bf k}_2,{\bf q}} J^{(j)}_{{\bf k}_1,{\bf q}} + 4J^{(j)}_{-{\bf q},{\bf q}} W^{(j)}_{{\bf q},-{\bf q},{\bf k}_1, {\bf k}_2}}{\omega - |\varepsilon_j| + J^{(j)}_{{\bf q}}/4 + W^{(j)}_{{\bf q}}/2}
\end{aligned}\end{equation}


\end{document}
