% !TEX TS-program = xelatex
\documentclass[twoside]{article}

\usepackage{amsthm,amsmath,amssymb,braket,graphicx,enumitem,booktabs,multirow,booktabs,tcolorbox,wrapfig,cancel,caption,fancyhdr,relsize,textpos,booktabs,tocbibind,titlesec,geometry}
\usepackage[T1]{fontenc}
\usepackage[hidelinks,pdfusetitle]{hyperref}
\usepackage[stable]{footmisc}
\renewcommand{\arraystretch}{2}
\setlength\delimitershortfall{-2pt}
\pagestyle{fancy}
\fancyhf{}
\fancyfoot{}
\fancyfoot[C]{\thepage}
\fancyhead[OL]{\nouppercase\leftmark}    
\fancyhead[ER]{\rightmark}
\setlength{\headheight}{52pt}
\setcounter{tocdepth}{1}
\DeclareMathOperator{\sech}{sech}
\raggedbottom
\numberwithin{equation}{section}
\allowdisplaybreaks

\title{Generating Hubbard Model Solutions from Anderson Impurity Model Solutions}
\author{Abhirup Mukherjee, Dr. Siddhartha Lal}
\begin{document}
\maketitle
\section{Introduction}
This is an attempt to obtain solutions of the Hubbard model (groundstate wavefunction and ground state energy) using the solutions of the simpler single-impurity Anderson model (SIAM) (the models will be defined later). Broadly speaking, the method involves first solving the SIAM using a unitary renormalisation group approach, to get the ground state wavefunction and energy eigenvalue, and then combining these wavefunctions in a symmetrized fashion to get the wavefunction for the Hubbard model lattice. It is quite similar to dynamical mean-field theory (DMFT) in essence, but differs in practice. The similar part is the involvement of an impurity-solver. The difference, however, lies in the following points:
\begin{itemize}
	\item While DMFT primarily works with Green's functions and self-energies, this method involves Hamiltonians and wavefunctions.
	\item The impurity-solver in DMFT provides an impurity Green's function (which is then equated with the local Green's function of the bath), while the impurity-solver in this method actually provides a wavefunction.
	\item The final step of DMFT is the self-consistency equation, where the impurity and bath-local quantities are set equal. This ensures all sites, along with the impurity site, have the same self-energy, something which is required on grounds of  translational invariance. The present method, however, brings about the translational invariance in a different way. It symmetrizes the wavefunctions and Hamiltonians itself, such that all quantites then derived from the Hamiltonian or wavefunction are then guaranteed to have the symmetry.
\end{itemize}
The meaning of each of these statements will become clearer when we describe the method in more detail.

\end{document}
