% !TEX TS-program = xelatex
\documentclass{article}
\usepackage{amsthm,amsmath,amssymb,braket,graphicx,enumitem,booktabs,multirow,booktabs,tcolorbox,wrapfig,cancel,caption,fancyhdr,relsize,textpos,booktabs,tocbibind,titlesec,geometry}
\usepackage[T1]{fontenc}
\usepackage[hidelinks,pdfusetitle]{hyperref}
\usepackage[stable]{footmisc}
\renewcommand{\arraystretch}{2}
\setlength\delimitershortfall{-2pt}
\pagestyle{fancy}
\fancyhf{}
\fancyfoot{}
\fancyhead[L]{\nouppercase\leftmark}    
\fancyhead[R]{\thepage}    
\setlength{\headheight}{52pt}
\setcounter{tocdepth}{1}
\DeclareMathOperator{\sech}{sech}
\raggedbottom
\numberwithin{equation}{section}
\allowdisplaybreaks

\title{Generating Hubbard Model Solutions from Anderson Impurity Model Solutions}
\author{}%Abhirup Mukherjee, Dr. Siddhartha Lal}
\begin{document}
\maketitle
\section{Introduction}
This is an attempt to obtain solutions of the Hubbard model (groundstate wavefunction and ground state energy) using the solutions of the simpler single-impurity Anderson model (SIAM). The models are defined using the Hamiltonians
\begin{equation}\begin{aligned}
H_\text{hubb} &= -t^H\sum_{\sigma,\left<i,j \right>}\left(c^\dagger_{i\sigma} c_{j\sigma} + \text{h.c.}\right) + U^H\sum_i \hat n_{i \uparrow} \hat n_{i \downarrow} - \mu^H \sum_{i\sigma}\hat n_{i\sigma}\\
	H_\text{siam} &= -t^A\sum_{\sigma,\left<i,j \right>\atop{i \neq d \neq j}}\left(c^\dagger_{i\sigma} c_{j\sigma} + \text{h.c.}\right) + \epsilon_d^A \sum_\sigma\hat n_{d\sigma} + U^A \hat n_{d \uparrow} \hat n_{d \downarrow} - \mu^A \sum_{i\neq d,\sigma}\hat n_{i\sigma}
\end{aligned}\end{equation}
Broadly speaking, the method involves first solving the SIAM using a unitary renormalisation group approach, to get the ground state wavefunction and energy eigenvalue, and then combining these wavefunctions in a symmetrized fashion to get the wavefunction for the Hubbard model lattice. It is quite similar to dynamical mean-field theory (DMFT) in essence, but differs in practice. The similar part is the involvement of an impurity-solver. The difference, however, lies in the following points:
\begin{itemize}
	\item While DMFT primarily works with Green's functions and self-energies, this method involves Hamiltonians and wavefunctions.
	\item The impurity-solver in DMFT provides an impurity Green's function (which is then equated with the local Green's function of the bath), while the impurity-solver in this method actually provides a wavefunction.
	\item The final step of DMFT is the self-consistency equation, where the impurity and bath-local quantities are set equal. This ensures all sites, along with the impurity site, have the same self-energy, something which is required on grounds of  translational invariance. The present method, however, brings about the translational invariance in a different way. It symmetrizes the wavefunctions and Hamiltonians itself, such that all quantites then derived from the Hamiltonian or wavefunction are then guaranteed to have the symmetry.
\end{itemize}
The meaning of each of these statements will become clearer when we describe the method in more detail.

\section{Philosophy of the method}
The method is closely tied to the auxiliary system approach described in \cite{martin_2016}. We can view the full Hamiltonian as a sum of two component Hamiltonians \(H_1, H_2\) connected via the interaction term \(H_{12}\).
\begin{equation}\begin{aligned}
	H = \begin{pmatrix} H_1 && H_{12} \\ H_{12}^* && H_2 \end{pmatrix} = H_1 \ket{1}\bra{1} + H_2\ket{2}\bra{2} + H_{12}\ket{1}\bra{2} + H_{12}^*\ket{2}\bra{1}
\end{aligned}\end{equation}
where \(\ket{1(2)}\) actually represents a sum over all basis kets of the subsystem 1(2). As an example, we can split the the Hubbard model Hamiltonian between a particular site \(i = p\) and the rest of the lattice as follows:
\begin{equation}\begin{aligned}
	H_\text{hubb} &= \overbrace{U^H\hat n_{p \uparrow} \hat n_{p \downarrow} - \mu^H \sum_\sigma \hat n_{p \sigma}}^{H_1} \\
		      &+ \underbrace{U^H\sum_{i \neq p}\hat n_{i \uparrow} \hat n_{p \downarrow} - \mu^H \sum_{i \neq p, \sigma} \hat n_{i \sigma} -t^H\sum_{\sigma,\left<i,j \right>\atop{i \neq p \neq j}}\left(c^\dagger_{i\sigma} c_{j\sigma} + \text{h.c.}\right)}_{H_2}\\
		      & -\underbrace{t^H\sum_{\sigma,\atop{i \in \text{N.N. of }p}}\left(c^\dagger_{i\sigma} c_{p\sigma} + \text{h.c.}\right)}_{H_{12} + H_{12}^*}\\
\end{aligned}\end{equation}
The Green's function of the full Hamiltonian can also be split in a similar fashion:
\begin{equation}\begin{aligned}
	G(\omega) = \begin{pmatrix} G_1 && G_{12} \\ G_{12}^* && G_2 \end{pmatrix} 
\end{aligned}\end{equation}
The subsystem 1 is usually taken to be the "smaller system", and consequently, subsystem 2 represents the "bath". The smaller system is typically chosen such that its eigenstates are known exactly. Progress is then made by choosing a simpler version of the bath \(H_2\) and a simpler form also for its coupling \(H_{12}\) with the smaller system. This combination of the smaller system and the simpler bath is then called the \textit{auxiliary system}. A typical auxiliary system for the Hubbard model would be the SIAM, where the impurity represents an arbitrary site \(p\) of the lattice, the bath represents the rest of the lattice sites and the hybridisation term between the impurity and the bath represents the coupling term \(H_{12}\).


The algorithm of DMFT then involves starting with some local self-energy of the bath, \(\Sigma(\omega)\), and using an impurity solver to calculate the impurity Green's function in the presence of this self-energy. This impurity Green's function is then used to calculate the impurity self-energy \(\Sigma_d(\omega)\), and the self-energy of the bath is then set equal to this impurity self-energy: \(\Sigma(\omega) = \Sigma_d(\omega)\), because we expect, on grounds of the lattice symmetry, that the impurity is the same as any other site in the bath. This is said to be the self-consistency step, because the bath self-energy is completely determined only at the end. With this updated bath self-energy, one then repeats the entire process until there is no further change in the bath self-energy at the self-consistency step.


The present method intends to calculate the quantities in a different fashion. We start with a SIAM (possibly with a correlated bath having a non-trivial self-energy), and solve it using the unitary renormalisation group approach to get to a fixed-point Hamiltonian. At the fixed-point, assuming the couplings are much larger compared to the kinetic energy, we can approximate the Hamiltonian by a zero-mode of the kinetic energy part, which leads to a two-site Anderson molecule Hamiltonian. Such a model is exactly solvable. Armed with such a groundstate, we will then create the wavefunction of the entire Hubbard Hamiltonian by combining them according to a prescription which is guided by the symmetries of the problem. The impurity electron will be read off as an arbitrary site, \(p\), of the lattice, while the zero-mode site becomes the site that is nearest to it. There are also integrals of motion that come out of the URG, they make up the rest of the sites in the lattice. We will first demonstrate this explicitly for the Hubbard dimer. Then we will show how we can create a full Hubbard Hamiltonian by joining Anderson molecule Hamiltonians, and this should hopefully be a step towards getting a general prescription for generating wavefunctions for the full lattice (not just dimers).

\section{Solution of the Hubbard dimer using the Anderson molecule}
The Hubbard dimer and Anderson molecules (zero-mode) are defined by the following respective Hamiltonians:
\begin{equation}\begin{aligned}
	H^H &= -t^H\sum_{\sigma}\left(c^\dagger_{1\sigma}c_{2\sigma} + \text{h.c.}\right) + U^H\sum_{i=1,2}\hat n_{i \uparrow}\hat n_{i \downarrow} - \mu^H \sum_{\sigma, i=1,2}\hat n_{i\sigma}\\
	H^A &= -t^A\sum_{\sigma}\left(c^\dagger_{d\sigma}c_{z\sigma} + \text{h.c.}\right) + \epsilon_d^A \sum_{\sigma}\hat n_{d\sigma} + U^A\hat n_{d \uparrow}\hat n_{d \downarrow}
\end{aligned}\end{equation}
In the first Hamiltonian, the indices \(i=1,2\) refer to the two lattice sites that constitute the dimer. In the second Hamiltonian, the subscript \(d\) indicates the impurity site, while the subscript \(z\) indicates the zero-mode site. First, we will assume that the Hubbard dimer is at half-filling (\(\frac{1}{2}U^H = \mu^H\)):
\begin{equation}\begin{aligned}
	\label{hubb_dimer}
	H^H &= -t^H\sum_{\sigma}\left(c^\dagger_{1\sigma}c_{2\sigma} + \text{h.c.}\right) + U^H\sum_{i=1,2}\hat \tau_{i \uparrow}\hat \tau_{i \downarrow} + \left(\frac{1}{2}U^H- \mu^H\right) \sum_{\sigma, i=1,2}\hat n_{i\sigma} + \text{constant}\\
	    &= -t^H\sum_{\sigma}\left(c^\dagger_{1\sigma}c_{2\sigma} + \text{h.c.}\right) + U^H\sum_{i=1,2}\hat \tau_{i \uparrow}\hat \tau_{i \downarrow}
\end{aligned}\end{equation}
Since the Hubbard Hamiltonian is at half-filling, we will also place the impurity at half-filling by setting \(\epsilon_d^A = -\frac{1}{2}U^A\):
\begin{equation}\begin{aligned}
	\label{and_dimer}
	H^A &= -t^A\sum_{\sigma}\left(c^\dagger_{d\sigma}c_{z\sigma} + \text{h.c.}\right) + \left(\epsilon_d^A + \frac{1}{2}U^A\right) \sum_{\sigma}\hat \tau_{d\sigma} + U^A\hat \tau_{d \uparrow}\hat \tau_{d \downarrow} + \text{constant}\\
	    &= -t^A\sum_{\sigma}\left(c^\dagger_{d\sigma}c_{z\sigma} + \text{h.c.}\right) + U^A\hat \tau_{d \uparrow}\hat \tau_{d \downarrow}
\end{aligned}\end{equation}
The first step is to recreate the Hubbard dimer Hamiltonian eq.~\ref{hubb_dimer} using the Anderson molecule Hamiltonian eq.~\ref{and_dimer}:
\begin{equation}\begin{aligned}
	H^H &= -t^H\sum_{\sigma}\left(c^\dagger_{1\sigma}c_{2\sigma} + \text{h.c.}\right) + U^H\sum_{i=1,2}\hat \tau_{i \uparrow}\hat \tau_{i \downarrow}\\
	    &= \frac{1}{2}\left[-t^H\sum_{\sigma}\left(c^\dagger_{1\sigma}c_{2\sigma} + \text{h.c.}\right) + t^H\sum_{\sigma}\left(c^\dagger_{2\sigma}c_{1\sigma} + \text{h.c.}\right)\right] + \frac{1}{2} 2U^H\sum_{i=1,2}\hat \tau_{i \uparrow}\hat \tau_{i \downarrow}\\
	    &= \frac{1}{2}\left[-t^A\sum_{\sigma}\left(c^\dagger_{d\sigma}c_{z\sigma} + \text{h.c.}\right)\bigg\vert_{z \to 2, d \to 1\atop{t^A \to t^H}} + t^A\sum_{\sigma}\left(c^\dagger_{2\sigma}c_{1\sigma} + \text{h.c.}\right)\bigg\vert_{d \to 2,z \to 1\atop{t^A \to t^H}} \right] \\
	    &+ \frac{1}{2} \left(U^A\hat \tau_{i \uparrow}\hat \tau_{i \downarrow}\bigg\vert_{d \to 1\atop{U^A \to 2U^H}} + U^A\hat \tau_{i \uparrow}\hat \tau_{i \downarrow}\bigg\vert_{d \to 2\atop{U^A \to 2U^H}}\right)\\
	    &=\frac{1}{2}\left[H^A\left(t^A \to t^H, U^A \to 2U^H, d \to 1, z \to 2\right) + H^A\left(t^A \to t^H, U^A \to 2U^H, d \to 2, z \to 1\right)\right]
\end{aligned}\end{equation}
The conclusion we can draw from this is that the Hubbard dimer Hamiltonian can be obtained from the Anderson dimer Hamiltonian in the following fashion:
\begin{itemize}
	\item The essential idea is that we have to create a local Hubbard Hamiltonian for each site of the Hubbard lattice by replacing the impurity label \(d\) in the Anderson dimer with the label of the particular site. So if there are two sites, we will get two local Hamiltonians obtained by replacing \(d\) with 1 and 2 respectively. For each local Hamiltonian, the zero-mode label \(z\) is replaced by the site that is nearest to the one that \(d\) is being replaced by. So, if \(d \to 1(2)\), then \(z \to 2(1)\).
	\item This, however, is not the only change that we must make, in order to get the local Hamiltonian for a particular site. Along with \(d\) and \(z\), we must also make the transformations \(t^A \to t^H, U^A \to 2U^H\).
	\item Finally, once we have the local Hamiltonians for sites 1 and 2, we average them to get the total Hubbard Hamiltonian.
\end{itemize}
Note that we expect most of these "rules" to be specific for the dimer, and there will be generalizations to most of them for a general \(N-\)site Hubbard Hamiltonian.


The wavefunctions for the \(N=2\) sector can also be connected through these transformations. Since both the Hamiltonians are analytically solvable, we can write down their groundstate wavefunctions \cite{pavarini}:
\begin{equation}\begin{aligned}
	\ket{\Psi^H_\text{GS}} &= a_1(-U^H, t^H)\frac{1}{\sqrt 2}\left(\ket{\uparrow_1, \downarrow_2} - \ket{\downarrow_1, \uparrow_2}\right) - a_2(-U^H,t^H){\sqrt 2}\left(\ket{\uparrow_1\downarrow_1, } - \ket{,\uparrow_2\downarrow_2}\right)\\
	\ket{\Psi^A_\text{GS}} &= a_1(-\frac{1}{2}U^A, t^A)\frac{1}{\sqrt 2}\left(\ket{\uparrow_d, \downarrow_z} - \ket{\downarrow_d, \uparrow_z}\right) - a_2(-\frac{1}{2}U^A,t^A){\sqrt 2}\left(\ket{\uparrow_d\downarrow_d, } - \ket{,\uparrow_z\downarrow_z}\right)\\
	E^H_\text{GS} &=  -\frac{1}{2}\Delta\left( U^H, t^H \right), E^A_\text{GS} =  -\frac{1}{2}\Delta\left( \frac{1}{2}U^A, t^A \right)
\end{aligned}\end{equation}
where 
\begin{equation}\begin{aligned}
	a_1(U,t) \equiv \frac{-4t}{\sqrt{2\Delta(U,t)\left( \Delta(U,t) + U \right) }}, && a_2(U,t) \equiv \sqrt{\frac{\Delta(U,t) + U}{2\Delta(U,t)}}, &&\Delta(U,t) \equiv \sqrt{U^2 + 16t^2}\\
\end{aligned}\end{equation}
$a_1,a_2$ satisfy $a_1(-U,t) = -a_2(U,t)$.
From the forms of the wavefunctions and eigenenergies, we can immediately write down
\begin{equation}\begin{aligned}
	\ket{\Psi^H_\text{GS}} &= \frac{1}{2}\left[\ket{\Psi^A_\text{GS}}\left(t^A, U^A \to t^H, 2U^H, d \to 1, z \to 2\right) + \ket{\Psi^A_\text{GS}}\left(t^A, U^A \to t^H, 2U^H, d \to 2, z \to 1\right)\right]\\
	E^H_\text{GS} &= E^A_\text{GS}\left(t^A, U^A \to t^H, 2U^H\right)
\end{aligned}\end{equation}
This shows that the rules laid out before work for the Hamiltonians, as well as the wavefunctions and energy eigenvalues of the \(N=2,0,4\) sector. These sectors specifically work because it is only in these sectors can we ensure that \(n_d = n_z\), which is required for the Hubbard Hamiltonian because \(n_1 = n_2\). In the other sectors (\(N=1,3\)), the impurity site and the zero-mode sites have to be singly-occupied in some part, and since the impurity site incurs a single-occupation cost of \(-\frac{U^H}{2}\) which is not borne by the zero-mode site, there is an intrinsic dissimilarity between the two sites of the Anderson molecule in this regime. This dissimilarity does not exist for the Hubbard model, so we cannot hope to connect the two models in this regime.

\section{Creating General \(N-\)site Hubbard Hamiltonian from Anderson molecule}
We will follow the strategy outlined in the previous section. For concreteness, we will consider a lattice of \(N\) lattice sites and \(w\) nearest neighbours for each site. Note that a uniform number of nearest neighbours means that there is perfect translational invariance on the lattice, which means there cannot be any edge sites. This is achieved by applying periodic boundary conditions on the edges of the lattice. A square 2d lattice is thus placed on a torus.


For each site \(i\) out of the \(N\) sites (having nearest neighbours \(\{i_j, j\in\left[1,w\right]\}\)), we will create \(w\) local Hamiltonians \(\{H^H_{i,i_j}, i\in \left[1,N\right], j \in \left[1,w\right]  \}\) out of the Anderson molecule Hamiltonian, by setting the impurity index \(d\) to \(i\) and the zero-mode index \(z\) to \(i_j\). We will also need to suitably transform the Hamiltonian parameters \(U^A, t^A\), but we will figure those out as we go along. All \(N\) sites will then produce \(Nw\) local Hamiltonians in total, and the general Hubbard Hamiltonian should be the average of these local Hamiltonians. 


The local Hamiltonians look like
\begin{equation}\begin{aligned}
	H^A_{i,i_j} = -t^A\sum_{\sigma}\left(c^\dagger_{i\sigma}c_{i_j\sigma} + \text{h.c.}\right) + U^A\hat \tau_{i \uparrow}\hat \tau_{i \downarrow}
\end{aligned}\end{equation}
We havent transformed the Anderson parameters \(U^A, t^A\) to the Hubbard parameters yet. The sum of all local Hamiltonians for a particular site \(i\) then gives the total Hamiltonian for that site:
\begin{equation}\begin{aligned}
	H^A_{i} = \sum_{j\in\left[1,w\right]}H^H_{i,i_j} = \sum_{\sigma\atop{j \in \text{N.N. of }i}}-t^A\left(c^\dagger_{i\sigma}c_{j\sigma} + \text{h.c.}\right) + w U^A\hat \tau_{i \uparrow}\hat \tau_{i \downarrow}
\end{aligned}\end{equation}
If we now sum over all the lattice sites and take the average of all the Hamiltonians (\(Nw\) in number), we get
\begin{equation}\begin{aligned}
	H^H = \frac{1}{Nw}\sum_{i=1}^N H^H_{i} = -\frac{2}{Nw}t^A\sum_{\sigma\atop{\left<ij \right>}}\left(c^\dagger_{i\sigma}c_{j\sigma} + \text{h.c.}\right) + \frac{1}{N}U^A\hat \tau_{i \uparrow}\hat \tau_{i \downarrow}
\end{aligned}\end{equation}
The factor of 2 in front of \(t^A\) appears because each nearest neighbour pair \(\left<i,j \right>\) is counted twice, once when \(d\to i\) and again when \(d \to j\). Comparing with the actual Hubbard model, we can guess the following transformations for the model paramters:
\begin{equation}\begin{aligned}
	\frac{2}{Nw}t^A \to t^H, && \frac{1}{N}U^A \to U^H
\end{aligned}\end{equation}
For the dimer versions, we had \(N=2, w=1\) and if we substitute these values into the transformations we recover the transformations we had discovered that worked for the dimer. The Hubbard Hamiltonian can thus be obtained from the Anderson dimer using the following equation
\begin{equation}\begin{aligned}
	H^H\left(t^H, U^H\right)  = \frac{1}{Nw}\sum_{\left<i,j\right>}\left[H^A\left(d \to i, z \to j\right) + H^A\left(d \to j, z \to i\right)\right]_{t^A \to \frac{Nw}{2}t^H \atop{U^A \to N U^H}}
\end{aligned}\end{equation}
In terms of the Green's function, this equation becomes
\begin{equation}\begin{aligned}
	\label{green_eq}
	\omega - {G_H}^{-1}(\omega) = \frac{1}{Nw}\sum_{\left<i,j\right>}\left[\left(\omega - {G_A}^{-1}(\omega)\right)\left(d, z \to i,j\right) + \left(\omega - {G_A}^{-1}(\omega)\right)\left(d,z \to j,i\right)\right]_{t^A \to \frac{Nw}{2}t^H \atop{U^A \to N U^H}}
\end{aligned}\end{equation}
where $G_H$ and $G_A$ are the Hubbard and SIAM Green's functions:
\begin{equation}\begin{aligned}
	G_H(\omega) = \frac{1}{\omega - H_H}, && G_A(\omega) = \frac{1}{\omega - H_A}
\end{aligned}\end{equation}
Since the two $\omega$ on the RHS of eq.~\ref{green_eq} is independent of the summation indices, they can be pulled out along with a factor. Each will come with a factor which is just the total number of nearest neighbour pairs, which is $ \frac{Nw}{2}$. This allows them to cancel the $\omega$ on the LHS. The equation then simplifies to
\begin{equation}\begin{aligned}
	\label{green_eq_final}
	{G_H}^{-1}(\omega) = \frac{1}{Nw}\sum_{\left<i,j\right>}\left[\tilde G_A^{-1}(\omega, i, j) + \tilde G_A^{-1}(\omega, j, i)\right]
\end{aligned}\end{equation}
where $\tilde G_A^{-1}(\omega, i, j) = \omega - \tilde H^A(i, j)$, and $H^A(i, j)$ is defined as
\begin{equation}\begin{aligned}
	\tilde H^A(i, j) = -\frac{Nw}{2}t^H\sum_{\sigma}\left(c^\dagger_{i\sigma}c_{j\sigma} + \text{h.c.}\right) + N U^H\hat \tau_{i \uparrow}\hat \tau_{i \downarrow}
\end{aligned}\end{equation}
$\tilde H^A(i, j)$ is nothing but the Anderson molecule Hamiltonian, but with the impurity replaced by the $i^\text{th}$ site and the zero-mode replaced by the $j^\text{th}$ site. It is important to note that $\tilde G_A^{-1}(\omega, i, j)$ only scatters between the sites $i,j$.


We will now calculate the single-particle site diagonal and site offdiagonal Green's functions. First lets consider the diagonal Green's function at $i^\text{th}$ site. The only terms that will contribute on the RHS of eq.~\ref{green_eq_final} are those that have $i$ in one of the indices. There will be $w$ such terms, because $i$ has $w$ nearest neighbours. Each such pair will have two terms, one with $i$ as the impurity and another with $i$ as the zero-mode. For the term that has $i$ as the impurity, the Green's function would then represent the impurity-diagonal Green's function $(G_{dd}^\sigma)$ of the zero-mode SIAM, while for the other term, the Green's function would represent the bath-diagonal or the zero-mode diagonal Green's function $(G_{zz}^\sigma)$.
\begin{equation}\begin{aligned}
	\left(G_{H}^{-1}(\omega)\right)_{i\sigma,i\sigma} = \frac{1}{N}\left[\left(\tilde G_{A}^{-1}(\omega)\right)_{d\sigma-d\sigma} + \left(\tilde G_{A}^{-1}(\omega)\right)_{z\sigma-z\sigma}\right]
\end{aligned}\end{equation}
Now we come to the off-diagonal Green's function for the nearest neighbour sites $i$ and $j$. This will receive contribution from only one nearest neighbour pair in the summation of the RHS of eq.~\ref{green_eq_final}, namely that of $\left(i,j\right)$. The two terms for this nearest neighbour pair will be one for impurity as $i$ and the other for impurity as $j$. The Greens function will thus be the impurity-zero mode and zero mode-impurity Green's functions:
\begin{equation}\begin{aligned}
	\left(G_{H}^{-1}(\omega)\right)_{i\sigma,j\sigma} &= \frac{1}{Nw}\left[\left(\tilde G_{A}^{-1}(\omega)\right)_{d\sigma-z\sigma} + \left(\tilde G_{A}^{-1}(\omega)\right)_{z\sigma-d\sigma}\right]
\end{aligned}\end{equation}

\newpage
\section*{Appendix}
\subsection*{Spectra of Hubbard dimer and Anderson molecule}
Here we document the spectra of the Hamiltonians in eqs.~\ref{hubb_dimer} and \ref{and_dimer}.
\begin{table}[htpb!]
	\centering
	\begin{tabular}{|c|c|c|}
	\hline
	eigenstate & symbol & eigenvalue \\
	\hline
	$\ket{0,0}$ & $\ket{0}$ & \( \frac{U^H}{2}\)\\
	$ \frac{1}{\sqrt 2}\left(\ket{\sigma,0} \pm \ket{0,\sigma}\right)$ & $\ket{0\sigma_\pm}$ & \(\mp t^H\)\\
	$\ket{\sigma,\sigma}$ & $\ket{\sigma\sigma}$ & \( -\frac{U^H}{2}\)\\
	$ \frac{1}{\sqrt 2}\left(\ket{\uparrow,\downarrow} + \ket{\downarrow,\uparrow}\right)$ & $\ket{ST}$ & \( -\frac{U^H}{2}\)\\
	$ \frac{1}{\sqrt 2}\left(\ket{2,0} - \ket{0,2}\right)$ & $\ket{CS}$ & \( \frac{U^H}{2}\)\\
	$a_1(\pm U^H, t^H) \frac{1}{\sqrt 2}\left(\ket{\uparrow,\downarrow} - \ket{\downarrow,\uparrow}\right) \pm a_2(\pm U^H, t^H)\frac{1}{\sqrt 2}\left(\ket{2,0} + \ket{0,2}\right)$ & $\ket{\pm}$ & \(\pm \frac{1}{2}\Delta(U^H, t^H)\)\\
	$ \frac{1}{\sqrt 2}\left(\ket{\sigma,2} \pm \ket{2,\sigma}\right)$ & $\ket{2\sigma_\pm}$ & \(\pm t^H\)\\
	$\ket{2,2}$ & $\ket{4}$ & \( \frac{U^H}{2}\)\\
\hline
	\end{tabular}
	\caption{Spectrum of Hubbard dimer at half-filling}
	\label{hubb_dim_spectrum}
	\begin{tabular}{|c|c|c|}
	\hline
	eigenstate & symbol & eigenvalue\\
	\hline
	\(\ket{0,0}\) & \(\ket{0}\) & \(\frac{U^A}{4}\)\\
	\(a_1\left(\pm U^A, t^A\right)\ket{\sigma,0} \pm a_2\left(\pm U^A, t^A\right)\ket{0, \sigma}\) & \(\ket{0\sigma_\pm}\) & \(\pm\frac{1}{4}\Delta(U^A, t^A)\)\\
	\(\ket{\sigma,\sigma}\) & \(\ket{\sigma\sigma}\) & \(-\frac{U^A}{4}\)\\
	\(\frac{1}{\sqrt 2}\left(\ket{\uparrow, \downarrow} + \ket{\downarrow, \uparrow}\right)\) & \(\ket{ST}\) & \(-\frac{U^A}{4}\)\\
	\(\frac{1}{\sqrt 2}\left(\ket{2, 0} - \ket{0, 2}\right)\) & \(\ket{CS}\) & \(\frac{U^A}{4}\)\\
	\(a_1\left(\pm \frac{U^A}{2}, t^A\right) \frac{1}{\sqrt 2}\left(\ket{\uparrow,\downarrow} - \ket{\downarrow, \uparrow}\right) \pm a_2\left(\pm \frac{U^A}{2}, t^A\right) \frac{1}{\sqrt 2}\left(\ket{2,0} + \ket{0, 2}\right)\) & \(\ket{\pm}\) & \(\pm\frac{1}{2}\Delta(\frac{U^A}{2}, t^A)\)\\
	\(-a_1\left(\pm U^A, t^A\right)\ket{\sigma,2} \pm a_2\left(\pm U^A, t^A\right)\ket{2, \sigma}\) & \(\ket{2\sigma_\pm}\) & \(\pm\frac{1}{4}\Delta(U^A, t^A)\)\\
	\(\ket{2,2}\) & \(\ket{4}\) & \(\frac{U^A}{4}\)\\
	\hline
	\end{tabular}
	\caption{Spectrum of Anderson molecule at particle-hole symmetry}
\end{table}	
\subsection*{Local Green's function for the Hubbard dimer}
From the spectral representation, we have the following expression for the local Green's function at the $i^\text{th}$:
\begin{equation}\begin{aligned}
	G_{i i}^\sigma(\omega) = \frac{1}{Z}\sum_{m,n}||\bra{m}c_{i\sigma}\ket{n}||^2\left( e^{-\beta E_m} + e^{-\beta E_n}\right)\frac{1}{\omega + E_m - E_n}
\end{aligned}\end{equation}
$m,n$ sum over the exact eigenstates. $E_m, E_n$ are the corresponding energies. We are interested in teh $T \to 0$ Green's function. In that limit, all exponentials except that for the ground state $E_{gs}$ will die out. The exponential inside the summation will then cancel the exponential in the partition function.
\begin{equation}\begin{aligned}
	G_{i i}^\sigma(\omega, T \to 0) &= \sum_{n}\left[||\bra{GS}c_{i\sigma}\ket{n}||^2\frac{1}{\omega + E_{GS} - E_n} + ||\bra{n}c_{i\sigma}\ket{GS}||^2\frac{1}{\omega + E_n - E_{GS}}\right]\\
					&= \sum_{n}\left[||\bra{n}c^\dagger_{i\sigma}\ket{GS}||^2\frac{1}{\omega + E_{GS} - E_n} + ||\bra{n}c_{i\sigma}\ket{GS}||^2\frac{1}{\omega + E_n - E_{GS}}\right]\\
\end{aligned}\end{equation}

The ground state $\ket{GS}$ is just the state $\ket{-}$ in the table \ref{hubb_dim_spectrum}. We will choose to look at $\sigma = \uparrow`$. Then,
\begin{equation}\begin{aligned}
	c_{1\uparrow}\ket{-} &= \frac{a_1}{\sqrt 2}\ket{0, \downarrow} - \frac{a_2}{\sqrt 2}\ket{\downarrow,0}\\
	c^\dagger_{1\uparrow}\ket{-} &= -\frac{a_1}{\sqrt 2}\ket{2, \uparrow} - \frac{a_2}{\sqrt 2}\ket{\uparrow,2}\\
\end{aligned}\end{equation}
The set of states $\ket{n}$ that give non-zero inner product $\ket{GS}$ are therefore
\begin{equation}\begin{aligned}
	\left\{ \ket{n} \right\} &= \ket{0\downarrow_\pm}\\
	||\bra{n}c_{\uparrow\sigma}\ket{GS}||^2 &= \frac{1}{4}\left(-a_2 \pm a_1 \right)^2 = \frac{1}{4}\left(1 \mp 2a_1a_2\right)\\
	\left\{ E_n \right\} &= \mp t
\end{aligned}\end{equation}
for the first inner product, and
\begin{equation}\begin{aligned}
	\left\{\ket{n}\right\} &= \ket{2\uparrow_\pm}\\
	||\bra{n}c^\dagger_{\uparrow\sigma}\ket{GS}||^2 &= \frac{1}{4}\left( a_2 \pm a_1 \right)^2 = \frac{1}{4}\left(1 \pm 2a_1a_2\right)\\
	\left\{ E_n \right\} &= \pm t
\end{aligned}\end{equation}
for the second. The Greens function is therefore
\begin{equation}\begin{aligned}
	G_{i i}^\uparrow(\omega, T \to 0) &= \frac{\omega}{2}\left[\frac{1 + 2a_1(-U^H, t^H)a_2(-U^H, t^H)}{\omega^2 - \left(E_{GS} + t\right)^2} + \frac{1 - 2a_1(-U^H, t^H)a_2(-U^H, t^H)}{\omega^2 - \left(E_{GS} - t\right)^2}\right]
\end{aligned}\end{equation}


\bibliographystyle{unsrt}
\bibliography{notes}
\end{document}
