Precisely at $W/J=(W/J)_{\text{PG}}$, the antinode joins this connected region of zeros in $J_{{\bf k}_1, {\bf k}_2}$, marking the decoupling of the antinodes from all other points in the neighbourhood of the FS. This is an interaction-driven Lifshitz transition of the FS, and marks the entry into a PG phase possessing Fermi arcs~\cite{WuFerrero2018}. Importantly, it coincides with an emergent two-channel Kondo (2CK) impurity model, where each channel corresponds to a pair of Fermi arcs on opposite faces of the conduction bath FS. The 2CK nature of the PG is guaranteed by the symmetry of $J_{{\bf k},{\bf k'}}$:

\vspace{0.25cm}
\par\noindent\textbf{\large Momentum-resolved Dynamical Spectral Weight Transfer}\\
Passage through the PG phase is accompanied by a highly structured transfer of spectral weight across the FS. Strong charge fluctuations develop between the nodal and antinodal regions of the FS in the PG regime of the impurity model (Fig.~\ref{chargeCorr} (a)), as captured by the correlator:
\begin{equation}
\chi_c({\bf k}_1, {\bf k}2) = \left\langle
c^{\dagger}_{{\bf k}1 \uparrow} c^{\dagger}_{{\bf k}1 \downarrow} c_{{\bf k}2 \downarrow} c_{{\bf k}_2 \uparrow} + \text{h.c.}\right\rangle~.
\end{equation}
These fluctuations dynamically redistribute low-energy spectral weight from the antinodes to higher energies, leading to selective gap formation. Accordingly, the Luttinger surfaces of the PG~\cite{dzyaloshinskii2003some} coincides with the appearance of poles of the lattice model self-energy $\Sigma ({\bf k},\omega=0)$ near the antinodes; these poles approach the nodes on tuning towards the Mott transition (Fig.~\ref{chargeCorr} (b)). This mirrors the coalescing of finite-frequency poles of the self-energy poles towards zero frequency in the underlying impurity model~(Fig.~\ref{chargeCorr} (c)). 

\begin{figure*}
    \centering
    \includegraphics[width=\textwidth]{fig5.pdf}
    \caption{(a): Imaginary part of impurity self-energy $\Sigma^{\prime\prime}(\omega)$ as a function of frequency $\omega$ for Fermi liquid (FL, $|W/J| < 1.79$) and pseudogapped phases (NFL, $|W/J| \geq 1.79$). $\Sigma''(\omega)$ falls to zero as $\omega\to 0$ for the FL, while it attains a peak in the pseudogap. All $\Sigma''(\omega)$ for the NFL are observed to lie above the Mott-Ioffe-Regel (MIR) bound (dashed blue line), while those for the FL lie below. (b): Scaling of $\Sigma^{\prime\prime}(\omega)$ with frequency for Fermi liquid and pseudogapped phases. The FL self-energy fits to $\Sigma^{\prime\prime} \sim \omega^{\alpha}$ with $\alpha\approx 2$, vanishing as $\omega\to 0$, while the NFL self-energy grows as $\Sigma^{\prime\prime} \approx a + b\omega^{\beta}$ for small $\omega$, with $\beta\approx 2$. Remarkably, the NFL exponent remains mostly unchanged through the entirety of the PG phase. (c): Variation of the zero-frequency imaginary self-energy $-\Sigma''(\omega=0)$ with $\omega$ in the FL and pseudogap phases (entry into the PG is marked by the vertical dashed line). The inset shows the same but in linear scale, in order to display the dramatic rise (by almost 30 times) on entering the PG.}
    \label{selfEnergy}
\end{figure*}

\vspace{0.25cm}
\par\noindent\textbf{\large Non-Fermi liquid excitations within the Pseudogap}\\
In the PG regime, the nature of gapless Fermi arcs changes dramatically. We have already argued that the low-energy dynamics of these gapless Fermi arcs are governed by an underlying two-channel Kondo (2CK) impurity model~\cite{Tsvelick_weigmann_mchannel_1985, emery_kivelson}. This is consistent with the rapid fall of the impurity quasiparticle residue $Z_\text{imp}$ (Fig.~\ref{channelDecoupling} (a)) from finite values in the FL phase to vanishingly small values just before the onset of the PG. Fig.~\ref{channelDecoupling} (b) shows the simultaneous emergence of increasingly uncompensated local magnetic moments upon traversing the PG phase. The accompanying impurity spectral function of the gapless arcs show a pseudogapped behaviour for $\omega\simeq 0$, with a rapid fall in the spectral weight at $\omega=0$ upon traversing the PG~(Fig.~\ref{channelDecoupling} (c)). The collapse of the Kondo resonance into a pseudogapped spectral function is accompanied by the redistribution of spectral weight~\cite{dzyaloshinskii2003some} in the impurity spectral function from $\omega\sim 0$ to the Hubbard sidebands at finite frequencies $\omega\simeq \pm 3$ (in units of the bandwidth) (Fig.~5 of \cite{suppmat}). Concomitant with this is the emergence of a zero-frequency peak in the imaginary part of the self-energy of the NFL in the PG phase, $-\Sigma''(\omega)\sim (a + \omega^{\beta})^{-1}$~, with $a$ being a constant~(Fig.\ref{selfEnergy} (a)). Remarkably, we find that the exponent $\beta=2$ characterises the NFL for the entire PG phase (see Fig.~\ref{selfEnergy}(b)), including the critical end-point. This is in stark contrast with the $\Sigma''(\omega)\sim \omega^{2}$ for the FL (Fig.~\ref{selfEnergy}(b)). 

\noindent
All $\Sigma''(\omega)$ for the NFL are observed to lie above the Mott-Ioffe-Regel (MIR) bound~\cite{GunnarssonRMP2003,Hussey2004} (dashed blue line in Fig.\ref{selfEnergy} (a)), while those for the FL lie below. The MIR bound is the maximum expected scattering rate in metals when the mean free path approaches the lattice spacing: 
$-2\Sigma^{\prime\prime}_\text{MIR} = 1/\tau_\text{MIR}~, ~\tau_\text{MIR} = l_\text{min}/v_{F}$~,where $l_\text{min}$ is the minimum mean free path in metals (equal to one lattice spacing) and $\tau_\text{MIR}$ is the associated lifetime of quasiparticles close to the FS (with Fermi velocity $v_F$). The transport lifetime $\tau (\omega)$ of the NFL thus undergoes a suppression as $\omega \to 0$, together with a shift of the Drude peak to finite $\omega$~\cite{Hussey2004,pustogow2021,qazilbash2008} (see Fig.~2 and Sec.V in \cite{suppmat}). The height of the zero-frequency peak rises by almost 4 orders of magnitude from the start of the PG till its end at the Mott transition point (Fig.\ref{selfEnergy}(c)); the dramatic growth of the peak height very near the Mott critical point coincides with the coalescing of the finite-frequency poles of the self-energy into a single pole at zero-frequency, signalling the singular nodal NFL present at the Mott quantum critical point.

\vspace{0.25cm}
\par\noindent\textbf{\large Non-local nature of the Pseudogap}\\
In Fig.~\ref{longranged}(a) and (b), the spin-flip correlations and mutual information between the impurity spin and conduction bath sites respectively are observed to undergo a crossover within the PG, from a short-ranged behaviour at its onset, to a long-ranged behaviour near the Mott transition. The entanglement is observed to be multipartite in nature: in Fig.~\ref{longranged}(c), the quantum Fisher information (QFI)~\cite{Hauke2016} computed for the ground state wavefunction using an operator corresponding to the sum of local spin-flip exchange processes shows a jump at the onset of the PG. Further, the FL is observed to possess bi-partite entanglement while the NFL of the PG phase displays 5-partite entanglement~\cite{balut2025,mazza2024}.\\

\noindent
These striking results imply that the Mott transition observed by us lies beyond the local quantum criticality scenario~\cite{Si2001}. Instead,  we observe the PG phase to be a novel state of strongly interacting quantum matter emergent from the breakdown of local Kondo screening. This state is described by a quantum critical Fermi surface with NFL Fermi arcs that display increasingly critical behaviour, i.e., dynamics described by non-local quantum fluctuations, and excitations that become truly long-ranged close to the transition.

\begin{figure*}
    \centering
    \includegraphics[width=\textwidth]{fig6.pdf}
    \caption{Spin-flip correlation $\braket{{\bf S}_d \cdot {\bf S}_r}$ (a) and mutual information $I_{2}(d,r)$ (b) between the impurity spin and conduction bath local spin density as a function of the distance {\bf r} between them, normalised against the value at $r=1$. Both decay very quickly in the FL phase (blue), but show long-ranged behaviour in the NFL phase (green and red), extending to the edges of the system at the critical point (red). (c): Evolution of the Quantum Fisher Information $F_Q$ for a nearest-neighbour spin-flip operator $\mathcal{O} = \sum_{i \in \text{odd}}(S_i^+S_{i+1}^- + \text{h.c.})$ through the first Lifshitz transition and the pseudogap. The vertical dashed line marks the onset of the PG. To the left of it, the QFI in the Fermi liquid phase shows at most bipartite entanglement ($F_Q < 2$ (below blue dashed line)), while the PG shows the presence of multipartite entanglement upto 5 parties ($F_Q > 4$ (above red dashed line)).}
    \label{longranged}
\end{figure*}

\vspace{0.25cm}
\par\noindent\textbf{\large Exactly solvable nodal Non-Fermi liquid at Mott Criticality}\\
Very close to the transition, the excitations of the nodal NFL correspond to those of a Hatsugai-Kohmoto model~\cite{Baskaran1991,Hatsugai1992}. This insight is obtained from a perturbation-theoretic treatment of the RG fixed point Hamiltonian of the impurity model for $W/J\lesssim (W/J)_{\text{PG}}$, by considering the effects of a small fixed point Kondo scattering probability \(J^*\) in the backdrop of a larger bath interaction parameter \(|W|\). This yields the HK model~\cite{Baskaran1991,Hatsugai1992} as the singular part of the effective Hamiltonian arising from forward scattering processes (see Sec.4 of \cite{suppmat}):
\begin{equation}\begin{aligned}\label{HKModel}
	\Delta \tilde H_{{\bf q}_1 = {\bf q}_2} = \sum_{{\bf q},\sigma}\epsilon_{{\bf q}}{n}_{{\bf q},\sigma} + u\sum_{{\bf q}, \sigma}n_{{\bf q} \sigma} n_{{\bf q} \bar\sigma}~,
\end{aligned}\end{equation}
where the number operator \(n_{{\bf q} \sigma} = \phi^\dagger_{{\bf q}, \sigma} \phi_{{\bf q}, \sigma}\) pertains to emergent fermionic relative modes $\phi_{{\bf q}, \sigma} = \frac{1}{\sqrt 2}\left(c_{{\bf N}_1 + {\bf q},\sigma} - c_{{\bf N}_1 + {\bf Q}_1 - {\bf q}, \sigma}\right)$ that are shifted by an excitation momentum \({\bf q}\) away from the nodal point \({\bf N}_1 = \left(\pi/2, \pi/2\right)\) and its 
partner \({\bf N}_1 + {\bf Q}_1 = \left(-\pi/2, -\pi/2\right)\). The kinetic energy \(\epsilon_{\bf q}\) and interaction energy \(u\sim J^{2}/W\) are dispersion and Kondo scales renormalised by conduction bath correlations (see Sec.4 of \cite{suppmat}).\\

\noindent
Consequently, the resulting NFL metal of the lattice model involves long-lived excitations of multiple \({\bf k}\)-states, and manifest in the form of a divergent one-particle self-energy at the non-interacting FS~\cite{Phillips2020}:
$\Sigma_{{\bf q}}(\omega) = -u^2/4(\omega - \epsilon_{\bf q})$~, such that $\Sigma_{\epsilon_{\bf q}=0}(\omega \to 0)  \to \infty$. This zero-frequency self-energy pole presages the transition into a Mott insulating phase, where it marks a hard gap in the spectral function for charge excitations. This is consistent with our findings for the lattice (Fig.~\ref{chargeCorr}(c)) and impurity self-energies~(Fig.\ref{selfEnergy} (c)). The nodal Mott metal thus comprises of a Greens function zero at the Fermi energy together with an anistropic massless Dirac dispersion, leading to a non-zero Chern number~\cite{vafekvishwanath2014,morimoto2016,calderon2025}. This topological index survives into the Mott insulator.\\

\noindent For small but non-zero values of \(\omega - \epsilon_{\bf k}\), we obtain a quasiparticle residue that vanishes with $\omega$, \(Z_\text{imp} \sim \omega^2/U^2 \). The scattering rate of this singular NFL possesses a sharp peak at the FS (\(\omega=\epsilon_{\bf k_{\mathrm{F}}}\)): \(\Gamma \sim U^2\delta(\omega - \epsilon_{\bf k})\), consistent with the sharply peaked Lorentzian $\Sigma''_{{\bf k}}(\omega)\sim \omega^{-2}$ captured in Fig.\ref{selfEnergy}(a). This quantum critical NFL metal is an example of a strongly coupled scale-invariant form of quantum matter. The exact solution for eq.\eqref{HKModel} reveals the presence of low-energy excitations comprised of holons and doublons~\cite{Hatsugai1992}. These features point to the nodal NFL as a long-ranged and multipartite entangled, strongly interacting scale-invariant state of quantum matter 
~\cite{Georgi2007PRL,Georgi2007,Phillips2013Unparticles,PhillipsLectures2014} that are completely disconnected from the quasiparticles of the FL. Following the arguments laid out in~\cite{phillips2022}, at finite temperatures, such a scale-invariant highly entangled non-Fermi liquid is likely associated with Planckian dissipation (i.e., $\Sigma'' (T)\sim k_{B}T$) and a resistivity that varies linearly with temperature ($\rho\sim T$)~\cite{zaanen2019,legros2019}. Additionally, we observe that the nodal metal possesses pairing fluctuations (see Sec.4 of \cite{suppmat}) that can become dominant upon doping~\cite{Phillips2020}.
\vspace{0.25cm}
\par\noindent\textbf{\large Pseudogap as a strongly coupled phase of quantum matter}\\
We now unveil an organising principle that leads to the remarkable properties observed above for the strongly interacting NFL of the PG phase. The existence of a sharp connected FS at $T=0$ can be understood as the existence of a topologically protected manifold of gapless chiral excitations in ${\bf k}$-space at the FS~\cite{Heath_2020}. The FS is characterised by a topological index corresponding to an anomaly in the quantum many-body theory for electrons, and can be understood as a generalised symmetry of such a system~\cite{lanave2025,McGreevy2023}. A theorem by Luttinger and Ward~\cite{luttinger1960ground} shows that a count of the physical charge (known as Luttinger's volume) is identical to the topological index (a so-called homotopy charge known as Luttinger's count) even in the presence of electronic interactions that do not disturb the FS. We will now argue that the emergence of antinodal Luttinger surfaces involve a disconnection of the FS (into Fermi arcs) and that, by following La Nave et al.~\cite{lanave2025}, the accompanying change in its topological properties leads to the existence of gapless NFL excitations that are non-local in nature.\\

\noindent
The antinodal Luttinger surfaces arise from the splitting of double poles of the single-particle Greens function on the FS into poles lying on opposite complex half-planes, together with zeros that are pinned at the FS. These changes in the analytic structure of the single-particle Greens function have important consequences. First, the emergent zeros break a $\mathbb{Z}_{2}$ symmetry of the FS~\cite{Anderson2001,Huang2022,lanave2025}. This symmetry is guaranteed within FL theory because the presence of quasiparticles vanishingly close to the FS allows the interchange of spin and charge degrees of freedom, promoting the separate $SU(2)$ symmetries of spin and charge to the larger symmetry of $O(4) = SU(2) \times SU(2) \times \mathbb{Z}_2$~\cite{Anderson2001}. The insertion of zeros of the Greens function at the Fermi surface in the PG, and the associated splitting of the Greens function, breaks this $\mathbb{Z}_{2}$ symmetry by placing the pole for the spin excitation in one half of the complex plane while placing that for the charge excitation in the other half~\cite{su2025anomalies,Huang2022}.

\noindent
Second, they signal a divergent electronic self-energy as a function of the wavevector ${\bf k}$, render ill-behaved the Luttinger-Ward functional of the interacting electronic problem, and violate the generalised symmetry encoded within it. The changes in the pole structure change the Luttinger count topological invariant, while the zeros give rise to an additional topological contribution (linked to the Adler-Bell-Jackiw-type chiral anomaly)~\cite{adler1969axial,bell1969pcac,Altshuler_1998,calderon2025}. As a consequence of the half-filled particle-hole symmetric nature of the system at hand, Luttinger's volume is preserved upon taking into account topological contributions from both the Luttinger count {\it and} the zeros~\cite{seki2017topological}. Importantly, La Nave et al.~\cite{lanave2025} argue, following recent developments in understanding generalised symmetries~\cite{Casini2021Symmetries}, that the additional anomaly arising from the Luttinger surfaces guarantees the existence of gapless NFL excitations that are non-local in nature.  

Thus, the drastic change in nature of the real-space excitations - from locally well-defined Landau quasiparticles of the FL to the increasingly nonlocal excitations of the NFL Fermi arcs in the PG phase - appears to be dictated by a topological principle and, therefore, robust under renormalisation. This signals the NFL Fermi arcs of the PG as an emergent phase of strongly interacting quantum matter - which we dub the {\it Mott metal} - that is the parent metal of the MI. The same topological principle connects the nonlocal unparticle-like gapless excitations~\cite{Georgi2007PRL,Georgi2007,Phillips2013Unparticles,PhillipsLectures2014} of the scale-invariant nodal NFL of the HK model observed precisely at the Mott critical point to those of the rest of 
PG phase, e.g., a universal scaling of $\Sigma''_{{\bf k}}(\omega)\sim (a+b\omega^{2})^{-1}$ of the NFL throughout the PG phase (Fig.~\ref{selfEnergy} (a)) and whose $\omega=0$ value continues to grow upon violating the MIR bound. 

\section{Conclusions}
Our analysis reveals that the Mott transition proceeds via a continuous evolution through a PG regime characterized by a singular NFL metal - the Mott metal - with deconfined holon–doublon excitations confined to nodal Fermi arcs. 
As the system approaches criticality, this metallic phase exhibits increasingly non-local correlations and a divergent self-energy, signaling the breakdown of Landau quasiparticles and the onset of long-range quantum entanglement. 
Anchored in two-channel Kondo dynamics at intermediate scales and governed by Hatsugai–Kohmoto physics near the critical endpoint, the Mott metal provides a unified framework for understanding anomalous metallicity in strongly correlated systems. Its fate under finite doping presents a compelling direction for future investigation.
